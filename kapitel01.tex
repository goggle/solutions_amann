\chapter{Grundlagen}
\section{Logische Grundbegriffe}
\section{Mengen}
\section{Abbildungen}
\section{Relationen und Verknüpfungen}
\section{Die natürlichen Zahlen}
\setcounter{aufgabe}{1}
\begin{aufgabe}
Folgende Identitäten sind durch vollständige Induktion zu verifizieren:
\begin{enumerate}
	\item[(a)] $\displaystyle \sum_{k=0}^n k = \frac{n (n+1)}{2}  , \; n \in \mathbb N $
	\item[(b)] $\displaystyle \sum_{k=0}^n k^2 = \frac{ n (n+1) (2n+1) }{6}, \; n \in \mathbb N$
\end{enumerate}
\end{aufgabe}
\begin{proof}
	\begin{enumerate}
		\item[(a)] Für $n = 0$ ist die Behauptung klar. Nach Induktionsannahme gelte
			\[
				\sum_{k=0}^{n-1} k = \frac{(n-1) n}{2} \ .
			\]
			Also folgt
			\[
				\sum_{k=0}^n k = \sum_{k=0}^{n-1} k + n = \frac{(n-1) n}{2} + n
					= \frac{n^2 - n + 2n}{2} = \frac{n^2 + n}{2} = \frac{n (n+1)}{2} \ .
			\]
		\item[(b)] Für $n = 0$ ist die Behauptung wieder klar.
			Sei nach Induktionsannahme
			\[
				\sum_{k=0}^{n-1} k^2 = \frac{(n-1) n (2(n-1) + 1)}{6} = \frac{(n-1) n (2n-1)}{6}
			\]
			Also folgt
			\begin{align*}
				\sum_{k=0}^n k^2 &= \sum_{k=0}^{n-1} + n^2
					= \frac{(n-1) n (2n-1)}{6} + \frac{6n^2}{6}
					= \frac{(n^2-n) (2n-1) + 6n^2}{6} \\
					&= \frac{2n^3 - n^2 - 2n^2 + n + 6n^2}{6}
					= \frac{2n^3 + 3n^2 + n}{6}
					= \frac{ n(2n^2 + 3n + 1)}{6} \\
					&= \frac{n (n+1) (2n+1)}{6} \ .
			\end{align*}
	\end{enumerate}
\end{proof}

\setcounter{aufgabe}{4}
\begin{aufgabe}
	\begin{enumerate}
		\item[(a)] Man verifiziere, dass für $n, m \in \mathbb N$ mit $m \leq n$ gilt:
			\[
				\left[ m! (n-m)! \right] \mid n!
			\]
		\item[(b)] Für $m, n \in \mathbb N$ werden die Binomialkoeffizienten $\binom n m \in \mathbb N$
			definiert durch
			\[
				\binom n m := \begin{cases} \frac{n!}{m! (n-m)!} \ , & n \leq m \\ 0 \ , & m > n \end{cases}
			\]
			Man beweise folgende Rechenregeln:
			\begin{enumerate}
				\item[(i)] $ \binom n m = \binom{n}{n-m} $
				\item[(ii)] $\binom{n}{m-1} + \binom{n}{m} = \binom{n+1}{m}, \; 1 \leq m \leq n$
				\item[(iii)] $\sum_{k=0}^n \binom n k = 2^n$
				\item[(iv)] $\sum_{k=0}^m \binom{n+k}{n} = \binom{n+m+1}{n+1} $
			\end{enumerate}
	\end{enumerate}
\end{aufgabe}
\begin{proof}
	\begin{enumerate}
		\item[(a)] Seien $n, m \in \mathbb N$, $m \leq n$.
			\[
				\frac{n!}{m! (n-m)!} = \frac{ n (n-1) \cdots (n-m+1)}{m!}
			\]
			Im Zähler haben wir $m$ Faktoren.
			% TODO: Argument funktioniert glaub nicht.
			Also gibt es einen Faktor $(n - i_1)$, so dass $m \mid (n-i_1)$.
			Setzen wir dieses Verfahren fort, so finden wir
			$m! \mid n (n-1) \cdots (n-m+1)$, also $m! (n-m)! \mid n!$.
		\item[(b)]
			\begin{enumerate}
				\item[(i)]
					\[
						\binom{n}{n-m} = \frac{n!}{(n-m)! \left( n - (n-m) \right)!}
							= \frac{n!}{(n-m)! m!} = \binom n m
					\]
				\item[(ii)]
					\begin{align*}
						\binom{n}{m-1} + \binom n m 
							&= \frac{n!}{(m-1)! (n-m+1)!} + \frac{n!}{m! (n-m)!} \\
							&= \frac{n! \cdot m}{m! (n-m+1)!} + \frac{n! \cdot (n-m+1)}{m! (n-m+1)!} \\
							&= \frac{n! (n+1)}{m! (n-m+1)!}
							= \frac{(n+1)!}{m! ( n+1-m)!}
							= \binom{n+1}{m}
					\end{align*}
				\item[(iii)]
					Wir beweisen die Behauptung mit vollständiger Induktion über $n$. Für $n=0$
					haben wir
					\[
						\binom 0 0 = \frac{0!}{0! \cdot 0!} = 1 = 2^0
					\]
					Nehmen wir an, es gelte $\sum_{k = 0}^{n-1} \binom{n-1}{k} = 2^{n-1}$, dann folgt mit 
					(ii):
					\begin{align*}
						\sum_{k=0}^n \binom n k 
							&= 1 + \sum_{k=1}^n \left[ \binom{n-1}{k-1} + \binom{n-1}{k} \right]
							= 1 + \sum_{k=0}^{n-1} \binom{n-1}{k} + \sum_{k=1}^n \binom{n-1}{k} \\
							&= 1 + 2^{n-1} + \sum_{k=1}^{n-1} \binom{n-1}{k}
							= 2^{n-1} + \sum_{k=0}^{n-1} \binom{n-1}{k}
							= 2^{n-1} + 2^{n-1} = 2^n
					\end{align*}
				\item[(iv)]
					Wiederum verwenden wir Induktion über $n$. Für $n=0$ haben wir 
					$\binom 0 0 = 1 = \binom{n+1}{n+1}$. Sei nach Induktionsannahme
					$\sum_{k=0}^{m-1} \binom{n+k}{n} = \binom{n+m}{n+1}$.
					Dann folgt
					\[
						\sum_{k=0}^m \binom{n+k}{n}
							= \sum_{k=0}^{m-1} \binom{n+k}{n} + \binom{n+m}{n}
							= \binom{n+m}{n+1} + \binom{n+m}{n}
							= \binom{n+m+}{n+1} \ .
					\]
			\end{enumerate}
	\end{enumerate}
\end{proof}

\section{Abzählbarkeit}
\section{Gruppen und Homomorphismen}
\section{Ringe, Körper und Polynome}
\begin{aufgabe}
	Es seien $a$ und $b$ kommutierende Elemente eines Ringes mit Eins und $n \in \mathbb N$.
	Man beweise:
	\begin{enumerate}
		\item[(a)] $\displaystyle a^{n+1} - b^{n+1} = (a-b) \sum_{j=0}^n a^j b^{n-j} $
		\item[(b)] $\displaystyle a^{n+1} - 1 = (a-1) \sum_{j=0}^n a^j$ 
	\end{enumerate}
\end{aufgabe}
\begin{proof}
	\begin{enumerate}
		\item[(a)]
			\begin{align*}
				(a-b) \sum_{j=0}^n a^j b^{n-1}
				&= \sum_{j=0}^n a^{j+1} b^{n-j} - \sum_{j=0}^n a^j b^{n+1-j} \\
				&= a^{n+1} + \sum_{j=0}^{n-1} a^{j+1} b^{n-j} - \sum_{j=1}^n a^j b^{n+1-j} - b^{n+1} \\
				&= a^{n+1} + \left( \sum_{j=0}^{n-1} a^{j+1} b^{n-j} - \sum_{j=0}^{n-1} a^{j+1} b^{n-j}
					\right) - b^{n+1} \\
				&= a^{n+1} - b^{n+1}
			\end{align*}
		\item[(b)]
			Setze $b=1$ in (a).
	\end{enumerate}
\end{proof}

\setcounter{aufgabe}{2}
\begin{aufgabe} Sei $K$ ein Körper. Dann ist $K[X]$ nullteilerfrei. \end{aufgabe}
\begin{proof}
	Angenommen, es existieren $0 \neq p = \sum_{k=0}^n p_k X^k \in K[X]$ und
	$0 \neq q = \sum_{k=0}^m q_k X^k \in K[X]$ mit $p q = 0$, d.h.
	\[
		p q = \left( \sum_{k=0}^n p_k X^k \right) \left( \sum_{k=0}^m q_k X^k \right)
		= \sum_{k=0}^{n+m} \underbrace{ \left( \sum_{\ell = 0}^k p_\ell q_{k-\ell} \right)}_{(pq)_k} X^k
			= 0
	\]
	Seien $i$ und $j$ die kleinsten Indizes mit $p_i \neq 0$ und $q_j \neq 0$. Dann ist aber
	\begin{align*}
		(pq)_{i+j} &= \sum_{\ell = 0}^{i+j} p_{\ell} q_{(i+j)-\ell} \\
		&= \underbrace{ p_0 }_{=0} q_{i+j} + \ldots
		+ \underbrace{p_{i-1}}_{=0} q_{j+1} 
		+ \underbrace{p_i q_j}_{\neq 0}
		+ p_{i+1} \underbrace{q_{j-1}}_{=0} + \ldots
		+ p_n \underbrace{ q_{0} }_{=0} \neq 0
	\end{align*}
	ein Widerspruch.
\end{proof}

\section{Die rationalen Zahlen}
\section{Die reellen Zahlen}
\setcounter{aufgabe}{5}
\begin{aufgabe}
	Man beweise die \textit{Bernoullische Ungleichung}: Für $x \in \mathbb R$
	und $n \in \mathbb N$ gilt
	\[
		(1 + x)^n \geq 1 + nx \ .
	\]
\end{aufgabe}
\begin{proof}
	Wir verwenden vollständige Induktion. Für $n = 0$ ist
	$(1 + x)^0 = 1 \geq 1 = 1 + 0 \cdot x$.

	Sei nach Induktionsannahme $(1 + x)^{n-1} \geq 1 + (n-1) x$.
	Dann folgt:
	\begin{align*}
		(1 + x)^{n-1}
			&= (1 + x) (1 + x)^{n-1}
			\geq (1 + x) \left( 1 + (n-1)x \right) \\
			&= 1 + \underbrace{ (n-1) x + x}_{=nx} + \underbrace{ (n-1) x^2}_{ \geq 0}
			\geq 1 + nx
	\end{align*}
\end{proof}

\setcounter{aufgabe}{9}
\begin{aufgabe}
	Es seien $n \in \mathbb N^\times$ und $x = (x_1, \ldots, x_n) \in \left[ \mathbb R^+ \right]^n$.
	Dann heisst $g(x) := \sqrt[n]{\prod_{j=1}^n x_j}$ bzw. $a(x) := \frac 1 n \sum_{j=1}^n x_j$
	\textbf{geometrisches} bzw. \textbf{arithmetisches Mittel} der $x_1, \ldots, x_n$. Zu beweisen
	ist die \textit{Ungleichung zwischen dem geometrischen und arithmetischen Mittel}, d.h.
	$g(x) \leq a(x)$.
\end{aufgabe}
\begin{proof}
Wir beweisen die Ungleichung per vollständige Induktion über $n$. Für $n = 1$ gilt
$g(x) = x_1 \leq \frac 1 1 x_1 = a(x)$.

Sei die Behauptung wahr für ein $n \in \mathbb N$. Wir können $x_i > 0$ annehmen für
$i = 1, \ldots n+1$, da sonst die Behauptung trivial ist. Seien also $x_1, \ldots, x_{n+1} 
\in \mathbb R_{>0}$ und ohne Einschränkung sei $x_{n+1} \geq x_i$ für $i = 1, \ldots, n$.
Dann ist
\[
	a(x_1, \ldots, x_n) = \frac{x_1, \ldots, x_n}{n}
		\leq \frac{n x_{n+1}}{n}
		= x_{n+1} \ .
\]
Also ist
\[
	y := \frac{ x_{n+1} - a(x_1, \ldots, x_n) }{(n+1) a(x_1, \ldots, x_n)} \geq 0
\]
und wegen
\begin{align*}
	1 + y
		&= \frac{ (n+1) a(x_1, \ldots, x_n) + x_{n+1} - a(x_1, \ldots, x_n)}{(n+1) a(x_1, \ldots, x_n)}
		= \frac{n a(x_1, \ldots, x_n) + x_{n+1}}{(n+1) a(x_1, \ldots, x_n)} \\
		&= \frac{n}{n+1} + \frac{x_{n+1}}{(n+1) a(x_1, \ldots, x_n)}
\end{align*}
folgt aus der Bernoulli-Ungleichung
\begin{align*}
	\left( \frac{x_1 + \ldots + x_{n+1}}{ (n+1) a(x_1, \ldots, x_n)} \right)^{n+1}
	&= (1 + y)^{n+1} \geq 1 + (n+1) y = \frac{x_{n+1}}{a(x_1, \ldots, x_n)} \ .
\end{align*}
Nun folgt mit der Induktionsannahme die Behauptung:
\begin{align*}
	%\frac{x_1 + \ldots + x_{n+1}}{n+1} 
	\frac{1}{n+1} \sum_{j=1}^{n+1} x_j
	&\geq a(x_1, \ldots, x_n)^{n+1} \frac{x_{n+1}}{a(x_1, \ldots, x_n)}
	= a(x_1, \ldots, x_n)^n x_{n+1} \\
	&\geq g(x_1, \ldots, x_n)^n x_{n+1} = \prod_{j=1}^{n+1} x_j
\end{align*}
\end{proof}

\begin{aufgabe}
	Für $x = (x_1, \ldots, x_n)$ und $y = (y_1, \ldots, y_n) \in \mathbb R^n$ sei 
	$x \centerdot y := \sum_{j=1}^n x_j y_j$. Man beweise folgende Ungleichung
	zwischen dem \textbf{gewichteten} \textit{geometrischen} und dem \textbf{gewichteten}
	\textit{arithmetischen Mittel}:
	\[
		\sqrt[|\alpha|]{x^\alpha} \leq \frac{x \centerdot \alpha}{|\alpha|}, 
		\qquad x \in \left[ \mathbb R^+ \right]^n, \; \alpha \in \mathbb N^n \ .
	\]
\end{aufgabe}
\begin{proof}
Sei $\alpha := ( \alpha_1, \dots, \alpha_n) \in \mathbb N^n$, 
$m := | \alpha | = \alpha_1 + \dots + \alpha_n$
und $x = (x_1, \dots, x_n) \in \left[ \mathbb R^+ \right]$. Es folgt
mit der Ungleichung zwischen dem geometrischen und arithmetischen Mittel:
\[
	\sqrt[|\alpha|]{x^\alpha}
	= \sqrt[m]{x_1^{\alpha_1} \cdots x_n^{\alpha_n}}
	\leq \frac 1 m \sum_{j=1}^n \alpha_j x_j
	= \frac{x \centerdot \alpha}{|\alpha|}
\]

\end{proof}

\section{Die komplexen Zahlen}
\section{Vektorräume, affine Räume und Algebren}
