\chapter{Grundlagen}
\section{Logische Grundbegriffe}
\section{Mengen}
\section{Abbildungen}
\section{Relationen und Verknüpfungen}
\section{Die natürlichen Zahlen}
\setcounter{aufgabe}{1}
\begin{aufgabe}
Folgende Identitäten sind durch vollständige Induktion zu verifizieren:
\begin{enumerate}
	\item[(a)] $\displaystyle \sum_{k=0}^n k = \frac{n (n+1)}{2}  , \; n \in \mathbb N $
	\item[(b)] $\displaystyle \sum_{k=0}^n k^2 = \frac{ n (n+1) (2n+1) }{6}, \; n \in \mathbb N$
\end{enumerate}
\end{aufgabe}
\begin{proof}
	\begin{enumerate}
		\item[(a)] Für $n = 0$ ist die Behauptung klar. Nach Induktionsannahme gelte
			\[
				\sum_{k=0}^{n-1} k = \frac{(n-1) n}{2} \ .
			\]
			Also folgt
			\[
				\sum_{k=0}^n k = \sum_{k=0}^{n-1} k + n = \frac{(n-1) n}{2} + n
					= \frac{n^2 - n + 2n}{2} = \frac{n^2 + n}{2} = \frac{n (n+1)}{2} \ .
			\]
		\item[(b)] Für $n = 0$ ist die Behauptung wieder klar.
			Sei nach Induktionsannahme
			\[
				\sum_{k=0}^{n-1} k^2 = \frac{(n-1) n (2(n-1) + 1)}{6} = \frac{(n-1) n (2n-1)}{6}
			\]
			Also folgt
			\begin{align*}
				\sum_{k=0}^n k^2 &= \sum_{k=0}^{n-1} + n^2
					= \frac{(n-1) n (2n-1)}{6} + \frac{6n^2}{6}
					= \frac{(n^2-n) (2n-1) + 6n^2}{6} \\
					&= \frac{2n^3 - n^2 - 2n^2 + n + 6n^2}{6}
					= \frac{2n^3 + 3n^2 + n}{6}
					= \frac{ n(2n^2 + 3n + 1)}{6} \\
					&= \frac{n (n+1) (2n+1)}{6} \ .
			\end{align*}
	\end{enumerate}
\end{proof}

\section{Abzählbarkeit}
\section{Gruppen und Homomorphismen}
\section{Ringe, Körper und Polynome}
\section{Die rationionalen Zahlen}
\section{Die reellen Zahlen}
\section{Die komplexen Zahlen}
\section{Vektorräume, affine Räume und Algebren}
