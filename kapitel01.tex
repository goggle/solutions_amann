\chapter{Grundlagen}
\section{Logische Grundbegriffe}
\section{Mengen}
\section{Abbildungen}
\section{Relationen und Verknüpfungen}
\section{Die natürlichen Zahlen}
\setcounter{aufgabe}{1}
\begin{aufgabe}
Folgende Identitäten sind durch vollständige Induktion zu verifizieren:
\begin{enumerate}
	\item[(a)] $\displaystyle \sum_{k=0}^n k = \frac{n (n+1)}{2}  , \; n \in \mathbb N $
	\item[(b)] $\displaystyle \sum_{k=0}^n k^2 = \frac{ n (n+1) (2n+1) }{6}, \; n \in \mathbb N$
\end{enumerate}
\end{aufgabe}
\begin{proof}
	\begin{enumerate}
		\item[(a)] Für $n = 0$ ist die Behauptung klar. Nach Induktionsannahme gelte
			\[
				\sum_{k=0}^{n-1} k = \frac{(n-1) n}{2} \ .
			\]
			Also folgt
			\[
				\sum_{k=0}^n k = \sum_{k=0}^{n-1} k + n = \frac{(n-1) n}{2} + n
					= \frac{n^2 - n + 2n}{2} = \frac{n^2 + n}{2} = \frac{n (n+1)}{2} \ .
			\]
		\item[(b)] Für $n = 0$ ist die Behauptung wieder klar.
			Sei nach Induktionsannahme
			\[
				\sum_{k=0}^{n-1} k^2 = \frac{(n-1) n (2(n-1) + 1)}{6} = \frac{(n-1) n (2n-1)}{6}
			\]
			Also folgt
			\begin{align*}
				\sum_{k=0}^n k^2 &= \sum_{k=0}^{n-1} + n^2
					= \frac{(n-1) n (2n-1)}{6} + \frac{6n^2}{6}
					= \frac{(n^2-n) (2n-1) + 6n^2}{6} \\
					&= \frac{2n^3 - n^2 - 2n^2 + n + 6n^2}{6}
					= \frac{2n^3 + 3n^2 + n}{6}
					= \frac{ n(2n^2 + 3n + 1)}{6} \\
					&= \frac{n (n+1) (2n+1)}{6} \ .
			\end{align*}
	\end{enumerate}
\end{proof}

\setcounter{aufgabe}{4}
\begin{aufgabe}
	\begin{enumerate}
		\item[(a)] Man verifiziere, dass für $n, m \in \mathbb N$ mit $m \leq n$ gilt:
			\[
				\left[ m! (n-m)! \right] \mid n!
			\]
		\item[(b)] Für $m, n \in \mathbb N$ werden die Binomialkoeffizienten $\binom n m \in \mathbb N$
			definiert durch
			\[
				\binom n m := \begin{cases} \frac{n!}{m! (n-m)!} \ , & n \leq m \\ 0 \ , & m > n \end{cases}
			\]
			Man beweise folgende Rechenregeln:
			\begin{enumerate}
				\item[(i)] $ \binom n m = \binom{n}{n-m} $
				\item[(ii)] $\binom{n}{m-1} + \binom{n}{m} = \binom{n+1}{m}, \; 1 \leq m \leq n$
				\item[(iii)] $\sum_{k=0}^n \binom n k = 2^n$
				\item[(iv)] $\sum_{k=0}^m \binom{n+k}{n} = \binom{n+m+1}{n+1} $
			\end{enumerate}
	\end{enumerate}
\end{aufgabe}
\begin{proof}
	\begin{enumerate}
		\item[(a)] Seien $n, m \in \mathbb N$, $m \leq n$.
			\[
				\frac{n!}{m! (n-m)!} = \frac{ n (n-1) \cdots (n-m+1)}{m!}
			\]
			Im Zähler haben wir $m$ Faktoren.
			% TODO: Argument funktioniert glaub nicht.
			Also gibt es einen Faktor $(n - i_1)$, so dass $m \mid (n-i_1)$.
			Setzen wir dieses Verfahren fort, so finden wir
			$m! \mid n (n-1) \cdots (n-m+1)$, also $m! (n-m)! \mid n!$.
		\item[(b)]
			\begin{enumerate}
				\item[(i)]
					\[
						\binom{n}{n-m} = \frac{n!}{(n-m)! \left( n - (n-m) \right)!}
							= \frac{n!}{(n-m)! m!} = \binom n m
					\]
				\item[(ii)]
					\begin{align*}
						\binom{n}{m-1} + \binom n m 
							&= \frac{n!}{(m-1)! (n-m+1)!} + \frac{n!}{m! (n-m)!} \\
							&= \frac{n! \cdot m}{m! (n-m+1)!} + \frac{n! \cdot (n-m+1)}{m! (n-m+1)!} \\
							&= \frac{n! (n+1)}{m! (n-m+1)!}
							= \frac{(n+1)!}{m! ( n+1-m)!}
							= \binom{n+1}{m}
					\end{align*}
				\item[(iii)]
					Wir beweisen die Behauptung mit vollständiger Induktion über $n$. Für $n=0$
					haben wir
					\[
						\binom 0 0 = \frac{0!}{0! \cdot 0!} = 1 = 2^0
					\]
					Nehmen wir an, es gelte $\sum_{k = 0}^{n-1} \binom{n-1}{k} = 2^{n-1}$, dann folgt mit 
					(ii):
					\begin{align*}
						\sum_{k=0}^n \binom n k 
							&= 1 + \sum_{k=1}^n \left[ \binom{n-1}{k-1} + \binom{n-1}{k} \right]
							= 1 + \sum_{k=0}^{n-1} \binom{n-1}{k} + \sum_{k=1}^n \binom{n-1}{k} \\
							&= 1 + 2^{n-1} + \sum_{k=1}^{n-1} \binom{n-1}{k}
							= 2^{n-1} + \sum_{k=0}^{n-1} \binom{n-1}{k}
							= 2^{n-1} + 2^{n-1} = 2^n
					\end{align*}
				\item[(iv)]
					Wiederum verwenden wir Induktion über $n$. Für $n=0$ haben wir 
					$\binom 0 0 = 1 = \binom{n+1}{n+1}$. Sei nach Induktionsannahme
					$\sum_{k=0}^{m-1} \binom{n+k}{n} = \binom{n+m}{n+1}$.
					Dann folgt
					\[
						\sum_{k=0}^m \binom{n+k}{n}
							= \sum_{k=0}^{m-1} \binom{n+k}{n} + \binom{n+m}{n}
							= \binom{n+m}{n+1} + \binom{n+m}{n}
							= \binom{n+m+}{n+1} \ .
					\]
			\end{enumerate}
	\end{enumerate}
\end{proof}

\section{Abzählbarkeit}
\section{Gruppen und Homomorphismen}
\section{Ringe, Körper und Polynome}
\section{Die rationionalen Zahlen}
\section{Die reellen Zahlen}
\section{Die komplexen Zahlen}
\section{Vektorräume, affine Räume und Algebren}
