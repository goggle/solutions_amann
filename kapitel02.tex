\chapter{Konvergenz}
\section{Konvergenz von Folgen}
\section{Das Rechnen mit Zahlenfolgen}
\section{Normierte Vektorräume}
\setcounter{aufgabe}{3}
\begin{aufgabe}
	Man beweise, dass in jedem Innenproduktraum $(E, (\cdot | \cdot))$ folgende
	\textbf{Parallelogrammidentität} gilt:
	\[
		2 ( \Vert x \Vert^2 + \Vert y \Vert^2)
			= \Vert x + y \Vert^2 + \Vert x - y \Vert^2, \quad x, y \in E
	\]
\end{aufgabe}
\begin{proof}
	\begin{align*}
		\Vert x + y \Vert^2 + \Vert x - y \Vert^2
			&= (x+y | x+y) + (x-y | x-y) \\
			&= (x | x+y) + (y |x+y) + (x | x-y) - (y | x-y) \\
			&= (x|x) + (x|y) + (y|x) + (y|y) + (x|x) - (x|y) - (y|x) + (y|y) \\
			&= 2 (x|x) + 2(y|y) \\
			&= 2 ( \Vert x \Vert^2 + \Vert y \Vert^2 )
	\end{align*}
\end{proof}

\setcounter{aufgabe}{5}
\begin{aufgabe}
	Es sei $(E, (\cdot | \cdot))$ ein reeller Innenproduktraum. Man beweise die Ungleichung
	\[
		(\Vert x \Vert + \Vert y \Vert) \frac{(x|y)}{\Vert x \Vert \Vert y \Vert}
			\leq \Vert x + y \Vert
			\leq \Vert x \Vert + \Vert y \Vert \ , \quad x,y \in E\setminus\{0\}
	\]
	Wann gilt Gleichheit?
\end{aufgabe}
\begin{proof}
	Die 2. Ungleichung ist gerade die Dreiecksungleichung. Hier ist also nichts zu beweisen.
	Die 1. Ungleichung ist trivial, falls $(x | y) \leq 0$, also können wir $(x | y) \geq 0$
	annehmen.
	Nach dem Quadrieren der 1. Ungleichung erhalten wir
	\[
		\left( \Vert x \Vert + \Vert y \Vert \right)^2 \frac{ (x|y)^2 }{\Vert x \Vert^2 \Vert y \Vert^2}
			\leq \Vert x + y \Vert^2 \ ,
	\]
	was zu zeigen ist. Nach der Cauchy-Schwarz-Ungleichung ist
	$\frac{ (x | y)^2}{ \Vert x \Vert^2 \Vert y \Vert^2} \leq 1$ und es
	folgt:
	\begin{align*}
		\left( \Vert x \Vert + \Vert y \Vert \right)^2 \frac{ (x|y)^2 }{\Vert x \Vert^2 \Vert y \Vert^2}
		 &= \left( \Vert x \Vert^2 + 2 \Vert x \Vert \Vert y \Vert + \Vert y \Vert^2 \right)
		 \frac{ (x | y)^2 }{ \Vert x \Vert^2 \Vert y \Vert^2} \\
		 &\leq \Vert x \Vert^2 + \Vert y \Vert^2 + 2 \frac{ (x | y) }{ \Vert x \Vert \Vert y \Vert}
		 (x | y) \\
		 &\leq \Vert x \Vert^2 + \Vert y \Vert^2 + 2 \frac{ \Vert x \Vert \Vert y \Vert}{ 
			 	\Vert x \Vert \Vert y \Vert } (x | y) \\
			&= \Vert x \Vert^2 + \Vert y \Vert^2 + 2 (x | y) \\
			&= (x | x) + (y | y) + 2(x | y) \\
			&= (x + y | x + y) = \Vert x + y \Vert^2
	\end{align*}
\end{proof}

\setcounter{aufgabe}{9}
\begin{aufgabe}
	Es sei $(E, (\cdot | \cdot))$ ein Innenproduktraum. Zwei Elemente $x, y \in E$ heissen
	\textbf{orthogonal}, wenn $(x | y) = 0$ gilt, man schreibt $x \perp y$. Eine Teilmenge
	$M \subset E$ heisst \textbf{Orthogonalsystem}, wenn $x \perp y$ für alle $x, y \in M$
	mit $x \neq y$ gilt. $M$ heisst \textbf{Orthonormalsystem}, falls $M$ ein 
	Orthogonalsystem ist mit $\Vert x \Vert = 1$ für $x \in M$.

	Es sei $\{x_0, \ldots, x_m\} \subset E$ ein Orthogonalsystem mit $x_j \neq 0$
	für $0 \leq j \leq m$. Man beweise:
	\begin{enumerate}
		\item[(a)] $\{x_0, \ldots, x_m\}$ ist linear unabhängig.
		\item[(b)] $\Big \Vert \displaystyle \sum_{k=0}^m x_k \Big \Vert^2 = \sum_{k=0}^m \Vert x_k \Vert^2$ 
			(Satz des Pythagoras)
	\end{enumerate}
\end{aufgabe}
\begin{proof}
	\begin{enumerate}
		\item[(a)] Angenommen $\{x_0, \ldots, x_m\}$ ist linear abhängig. Dann gibt es ein 
			$j \in \{0, \ldots, m\}$ so, dass $x_j = \displaystyle  \sum_{i=0 \atop i \neq j}^m \alpha_i x_i$ 
			mit $\alpha_i \in \mathbb K$, also ist 
			$x_j - \displaystyle \sum_{i=0 \atop i \neq j}^m \alpha_i x_i = 0$.
			Es folgt
			\begin{align*}
				0 &= (0 | x_j) = \Big ( x_j - \sum_{i=0 \atop i \neq j}^m \alpha_i x_i \Big \vert x_j \Big) \\
					&= ( x_j | x_j ) - \sum_{i=0 \atop i \neq j}^m \alpha_i \underbrace{ (x_i|x_j) }_{=0} 
					= (x_j | x_j) \ ,
			\end{align*}
			was bedeutet, dass $x_j = 0$ sein muss. Das ist ein Widerspruch zur Voraussetzung.
		\item[(b)] 
			\begin{align*}
				\Big \Vert \sum_{k=0}^m x_k \Big \Vert^2 
					&= \Big ( \sum_{k=0}^m x_k \Big \vert \sum_{j=0}^m x_j \Big )
					= \sum_{k=0}^m \Big( x_k \Big \vert \sum_{j=0}^m x_j \Big) \\
					&= \sum_{k=0}^m \sum_{j=0}^m \underbrace{ (x_k | x_j ) }_{=0 \text{ für } k \neq j} 
					= \sum_{k=0}^m (x_k | x_k) = \sum_{k=0}^m \Vert x_k \Vert^2
			\end{align*}
	\end{enumerate}
\end{proof}

\begin{aufgabe}
	Es sei $F$ ein Untervektorraum eines Innenproduktraumes $E$. Man beweise, dass das 
	\textbf{orthogonale Komplement} von $F$, d.h.
	\[
		F^\perp := \{ x \in E \ ; \ x \perp y, \ y \in F \} \ ,
	\]
	ein abgeschlossener Untervektorraum von $E$ ist.
\end{aufgabe}
\begin{proof}
	Wir zeigen zuerst, dass $F^\perp$ ein Untervektorraum von $E$ ist.
	\begin{itemize}
		\item Da $(0 | y) = 0$ für jedes $y \in F$, ist $0 \in F^\perp$.
		\item Seien $x_1, x_2 \in F^\perp$. Dann ist $(x_1 | y) = (x_2 | y) = 0$ für $y \in F$.
			Also ist auch $(x_1 + x_2 | y) = (x_1 | y ) + (x_2 | y) = 0$ für $y \in F$ und somit
			ist $x_1 + x_2 \in F^\perp$.
		\item Seien $\lambda \in \mathbb K$ und $x \in F^\perp$, dann ist 
			$(x | y) = 0$ für $y \in F$. Also ist auch $(\lambda x | y) = \lambda (x | y) = 0$
			für $y \in F$ und somit ist $\lambda x \in F^\perp$.
	\end{itemize}
	Es bleibt zu zeigen, dass $F^\perp$ abgeschlossen ist. Sei $(x_n)$ eine Folge in $F^\perp$, 
	die in $E$ konvergiert, d.h. $\lim x_n = x \in E$. 
	Es gilt also $(x_n | y) = 0$ für jedes $n \in \mathbb N$ und $y \in F$.
	Für $y \in F$ ist 
	\[
		(x | y) = (x - x_n | y ) + \underbrace{ (x_n | y)}_{=0} = (x - x_n | y) \ .
	\]
	Mit der Cauchy-Schwarz-Ungleichung folgt
	\[
		| (x | y) | = | (x-x_n | y) | \leq \Vert x - x_n \Vert \Vert y \Vert \ ,
	\]
	und da $\Vert x_n - x \Vert \to 0$ für $n \to \infty$ ist $(x|y)=0$ und somit
	$x \in F^\perp$, d.h. $F^\perp$ ist abgeschlossen.
\end{proof}

\begin{aufgabe}
	Es seien $B = \{ u_0, \ldots, u_m \}$ ein Orthonormalsystem im Innenproduktraum $(E, (\cdot | \cdot))$
	und $F := \spann(B)$. Ferner sei 
	\[
		p_F: E \to F, \quad x \to \sum_{k=0}^m (x | u_k) u_k \ .
	\]
	Man beweise:
	\begin{enumerate}
		\item[(a)] $x - p_F(x) \in F^\perp, \ x \in E$
	\end{enumerate}
\end{aufgabe}
\begin{proof}
	\begin{enumerate}
		\item[(a)] Sei $y \in F$, $x \in E$. Wegen $F = \spann(B)$ kann $y$ geschrieben werden
			als $y = \sum_{i=0}^m \alpha_i u_i$ für geeignete $\alpha_i \in \mathbb K$. Es folgt:
			\begin{align*}
				( x - p_F(x) | y) 
					&= \Big( x - \sum_{k=0}^m (x | u_k) u_k \Big \vert y \Big) \\
					&= \sum_{i=0}^m \alpha_i (x | u_i)
						- \sum_{k=0}^m (x | u_k) \Big( u_k \Big \vert \sum_{i=0}^m \alpha_i u_i \Big) \\
					&= \sum_{i=0}^m \alpha_i (x | u_i) - \sum_{k=0}^m (x | u_k) \sum_{j=0}^m \alpha_j (u_k|u_j) \\
					&= \sum_{i=0}^m \alpha_i (x | u_i) - \sum_{k=0}^m \alpha_k (x | u_k) = 0
			\end{align*}
	\end{enumerate}
\end{proof}



\section{Monotone Folgen}

\setcounter{aufgabe}{3}
\begin{aufgabe}
Für $a \in (0, \infty)$ definiere man die reelle Folge $(x_n)$ rekursiv durch $x_0 \geq a$ und
\[
x_{n+1} := \frac 1 2 \left( x_n + \frac{a}{x_n} \right) \ , \; n \in \mathbb N
\]
Man beweise, dass $(x_n)$ monoton fallend gegen $\sqrt{a}$ konvergiert.
\end{aufgabe}
\begin{proof}
Man weist einfach nach, dass $x_n > 0$ für $n \in \mathbb N$.
Für $n \in \mathbb N$ gilt:
\begin{align*}
x_{n+1}^2 &= \left( \frac 1 2 \left( x_n + \frac{a}{x_n} \right) \right)^2
	= \frac 1 4 \left( x_n^2 + 2a + \frac{a^2}{x_n^2} \right) \\
	&= \frac 1 4 \left( x_n^2 - 2a  + \frac{a^2}{x_n^2} \right) + a 
	= \frac 1 4 \left( x_n - \frac{a}{x_n} \right)^2 + a \geq a
\end{align*}

Wir weisen nach, dass $(x_n)$ monoton fallend ist:
\[
x_{n+1} - x_n = \frac 1 2 \left( x_n + \frac{a}{x_n} \right) - x_n
	= \frac{a}{2x_n} - \frac 1 2 x_n = \frac{a - x_n^2}{2 x_n} \leq 0
\]
Also ist die Folge $(x_n)$ nach unten beschränkt, monoton fallend und konvergiert somit.

Es bleibt noch zu zeigen, dass $\lim x_n = \sqrt{a}$.
Aus 
\[
x_{n+1} - \sqrt{a}
	= \frac 1 2 \left( x_n + \frac{a}{x_n} \right) - \sqrt{a}
	= \frac 1 2 \left( x_n - \sqrt{a} + \frac{a}{x_n} - \sqrt{a} \right)
	= \frac 1 2 \left( 1 - \frac{\sqrt{a}}{x_n} \right) \left( x_n - \sqrt{a} \right)
\]
folgt
\begin{align*}
| x_{n+1} - \sqrt{a} |
	&= \frac 1 2 \underbrace{ \left| 1 - \frac{\sqrt{a}}{x_n} \right| }_{\leq 1}
	\left| x_n - \sqrt{a} \right|
	\leq \frac 1 2 | x_n - \sqrt{a} |
	\leq \ldots \leq \left( \frac 1 2 \right)^{n+1} | x_0 - \sqrt{a} | \ ,
\end{align*}
woraus folgt, dass $x_n \to \sqrt{a}$ für $n \to \infty$.

\end{proof}


\setcounter{aufgabe}{6}
\begin{aufgabe}
\begin{enumerate}
\item[(a)] Man beweise folgende Fehlerabschätzung für $n \in \mathbb N^\times$:
\[
0 < e - \sum_{k=0}^n \frac{1}{k!} < \frac{1}{n n!}
\]
\item[(b)] Man beweise, dass $e$ eine irrationale Zahl ist.
\end{enumerate}
\end{aufgabe}
\begin{proof}
\begin{enumerate}
\item[(a)] Da $e = \sum_{k=0}^\infty \frac{1}{k!}$ ist die erste Ungleichung klar.

Sei $y_m := \sum_{k = n+1}^{n + m} \frac{1}{k!}$. Es gilt 
$y_m \to e - \sum_{k=0}^n$ für $m \to \infty$.
\begin{align*}
y_m &= \sum_{k=n+1}^{n+m} \frac{1}{k!}
	= \frac{1}{(n+1)!} \left[ 1 + \frac{1}{n+2} + \frac{1}{(n+2)(n+3)} + \ldots
	+ \frac{1}{(n+2) \cdots (n+m)} \right] \\
	&< \frac{1}{(n+1)!} \left[ 1 + \frac{1}{n+1} + \left( \frac{1}{n+1} \right)^2 + \ldots
	+ \left( \frac{1}{n+1} \right)^{m-2} \right] \\
	&< \frac{1}{(n+1)!} \sum_{k=0}^\infty \left( \frac{1}{n+1} \right)^k
	= \frac{1}{(n+1)!} \frac{1}{1 - \frac{1}{n+1}} 
	= \frac{1}{(n+1)!} \frac{n+1}{n} = \frac{1}{n n!}
\end{align*}
Also gilt auch die zweite Ungleichung.

\item[(b)] Angenommen $e$ ist rational, dann gibt es $p, \ n \in \mathbb N^\times$
mit $e = \frac p n$. Nach (a) gilt dann:
\[
0 < \frac p n - \sum_{k=0}^n \frac{1}{k!} < \frac{1}{n n!}
\]
Also ist
\[
0 < \underbrace{ n! p - n \sum_{k=0}^n \frac{n!}{k!} }_{\in \mathbb Z} < 1 \ .
\]
Das ist aber nicht möglich, da es keine ganze Zahl zwischen 0 und 1 gibt.
\end{enumerate}
\end{proof}

\begin{aufgabe}
Es sei $(x_n)$ rekursiv definiert durch
\[
x_0 := 1, \quad x_{n+1} := 1 + \frac{1}{x_n}, \qquad n \in \mathbb N .
\]
Man zeige, dass die Folge $(x_n)$ konvergiert und bestimme ihren Grenzwert.
\end{aufgabe}
\begin{proof}
Wir zeigen zuerst, dass $x_n > 1$ für $n \geq 1$.
Für $n = 1$ haben wir $x_1 = 1 + \frac 1 1 = 2$.
Nehmen wir an, es gilt $x_n > 1$, so folgt $x_{n+1} = 1 + \frac{1}{x_n} > 1$.
Unmittelbar aus der rekursiven Definition $x_{n+1} = 1 + \frac{1}{x_n}$ und
aus $x_n > 1$ für $n \geq 1$ folgt $x_n < 2$ für $n \geq 2$.
Für $n \geq 1$ gilt sogar $x_n \in \left[ 1.5, \  2 \right]$, da
\[
1.5 = 1 + \frac 1 2 \leq x_{n+1} = 1 + \frac{1}{x_n} \leq 1 + \frac 1 1 = 2 \ .
\]
Insbesondere ist die Folge beschränkt.

Als nächstes zeigen wir, dass die Teilfolge $(x_{2n})$ monoton wachsend ist. Da 
$x_2 = 1 + \frac{1}{x_1} = 1 + \frac{1}{2} = \frac{3}{2}$ ist $x_2 > x_0 = 1$.
Nun sei nach Induktionsannahme $x_{2n} \geq x_{2(n-1)}$.
\begin{align*}
x_{2(n+1)} - x_{2n}
	&= 1 + \frac{1}{x_{2n+1}} - \left( 1 + \frac{1}{x_{2n-1}} \right)
	= \frac{1}{1 + \frac{1}{x_{2n}}} - \frac{1}{1 + \frac{1}{x_{2(n-1)}}}
	= \frac{x_{2n}}{x_{2n}+1} - \frac{x_{2(n-1)}}{x_{2(n-1)} + 1} \\
	&= \frac{ x_{2n} (x_{2(n-1)} + 1) - x_{2(n-1)} (x_{2n} + 1)}{(x_{2n}+1)(x_{2(n-1)}+1)}
	= \frac{x_{2n} - x_{2(n-1)}}{(x_{2n}+1)(x_{2(n-1)}+1)} \geq 0
\end{align*}
Also ist $(x_{2n})$ eine konvergente Teilfolge von $(x_n)$.

Wir weisen nun nach, dass $(x_n)$ eine Cauchyfolge ist. Sei dazu $n \geq 1$ beliebig.
\begin{align*}
| x_{n+1} - x_n | 
	&= \left| 1 + \frac{1}{x_n} - \left( 1 + \frac{1}{x_{n-1}} \right) \right|
	= \left| \frac{ x_{n-1} - x_n}{x_n \cdot x_{n-1}} \right|
	\leq \frac 1 2 | x_{n-1} - x_n | \\
	&\leq \ldots \leq \left( \frac 1 2 \right)^{n-1} \cdot |x_2 - x_1 |
	= \left( \frac 1 2 \right)^n
\end{align*}

Für $m \geq n \geq 1$ erhalten wir
\begin{align*}
| x_m - x_n | 
	&= | x_m - x_{m-1} + x_{m-1} - x_{m-2} \pm \ldots - x_n | \\
	&\leq |x_m - x_{m-1} | + | x_{m-1} - x_{m-2} | + \ldots + |x_{n+1} - x_n| \\
	&\leq \left( \frac 1 2 \right)^{m-1} + \left( \frac 1 2 \right)^{m-2} + \ldots
	+ \left( \frac 1 2 \right)^n \\
	&= \left( \frac 1 2 \right)^n \cdot \sum_{k=0}^{m-1} \left( \frac 1 2 \right)^k 
	\leq \left( \frac 1 2 \right)^n \cdot 2
	= \left( \frac 1 2 \right)^{n-1}
\end{align*}

Somit finden wir zu jedem $\epsilon > 0$ ein $N \in \mathbb N$ so, dass
\[
|x_m - x_n| < \epsilon, \quad \text{für } m \geq n \geq N \ ,
\]
und $(x_n)$ ist eine Cauchyfolge, die eine konvergente Teilfolge besitzt, also selbst
konvergent.

Sei $g \in \left[ 1.5, \ 2 \right]$ der Grenzwert von $(x_n)$.
Mit den Grenzwertsätzen folgt nun
\[
g = \lim_{n \to \infty} x_n = \lim_{n \to \infty} x_{n+1}
	= \lim_{n \to \infty} 1 + \frac{1}{x_n} = 1 + \frac 1 g
\]
Diese Gleichung hat die positive Lösung $g = \frac{{1} + \sqrt{5}}{2}$.
\end{proof}

\begin{proof}[Alternativer Beweis]
Wir zeigen per Induktion $|x_n - g| \leq \frac{1}{g^{n+1}}$, wobei $g$ die positive Lösung der Gleichung
$g = 1 + \frac 1 g $ bezeichnet.

Für $n = 0$ haben wir $|x_0 - g| = | 1 - g | = \left| - \frac 1 g \right| \leq \frac{1}{g^1}$.

Sei nach Induktionsannahme $|x_{n-1} - g | \leq \frac{1}{g^n}$. Dann folgt wegen
$x_n \geq 1$ für alle $n \in \mathbb N$:
\begin{align*}
| x_n - g | 
	&= \left| 1 + \frac{1}{x_{n-1}} - \left( 1 + \frac 1 g \right) \right|
	= \left| \frac{1}{x_{n-1}} - \frac 1 g \right| \\
	&= \left| \frac{g - x_{n-1}}{x_{n-1} \cdot g} \right|
	\leq \frac 1 g \cdot | x_{n-1} - g|
	\leq \frac 1 g \cdot \frac{1}{g^n} = \frac{1}{g^{n+1}}
\end{align*}
Da $g > 1$, folgt $x_n \to g$.
\end{proof}

\begin{aufgabe}
Die \textbf{Fibonacci-Zahlen} $f_n$ sind rekursiv definiert durch
\[
f_0 := 0 \ , \quad f_1 := 1 \ , \quad f_{n+1} := f_n + f_{n-1} \ , \; n \in \mathbb N^\times
\]
Man beweise, dass $\lim \left( \frac{ f_{n+1}}{f_n} \right) = g$, wobei $g$ der
Grenzwert aus Aufgabe 8 bezeichne.
\end{aufgabe}
\begin{proof}
Die Folge der Fibonacci-Zahlen ist monoton wachsend und für $n \geq 1$ gilt $f_n \geq 1$.
Sei $g$ der Grenzwert aus Aufgabe 8, also die positive Lösung der quadratischen Gleichung
$g = 1 + \frac 1 g$.
Sei $F_n := \frac{f_{n+1}}{f_n}$, $n \in \mathbb N^\times$. Wir wollen beweisen, dass
die Folge $(F_n)_{n \geq 1}$ den Grenzwert $g$ hat:
\begin{align*}
| F_n - g |
	&= \left| \frac{ f_{n+1}}{f_n} - g \right|
	= \left| \frac{f_n + f_{n-1}}{f_n} - g \right|
	= \left| 1 + \frac{1}{F_{n-1}} - \left( 1 + \frac 1 g \right) \right| \\
	&= \left| \frac{1}{F_{n-1}} - \frac 1 g \right|
	= \left| \frac{g - F_{n-1}}{ F_{n-1} \cdot g } \right|
	\leq \frac 1 g | F_{n-1} - g | 
	\leq \ldots \leq \left( \frac 1 g \right)^{n-1} |F_1 - g|
\end{align*}
Da $0 < \frac 1 g = g -1 < 1$ folgt
$\left( \frac 1 g \right)^n \to 0$ für $n \to \infty$ und damit
$\lim F_n = g$.
\end{proof}

\begin{aufgabe}
Es seien
\[
x_0 := 5 \ , \quad x_1 := 1 \ , \quad  x_{n+1} := \frac 3 2 x_{n} + \frac 1 3 x_{n-1} \ ,
 \; n \in \mathbb N^\times \ .
\]
Man verifiziere, dass $(x_n)$ konvergiert und bestimme $\lim x_n$.
\end{aufgabe}
\begin{proof}
Für $n \geq 1$ gilt:
\begin{align*}
| x_n - x_{n-1} | &= \left| \frac 2 3 x_{n-1} + \frac 1 3 x_{n-2} - x_{n-1} \right|
	= \frac 1 3 | x_{n-1} - x_{n-2} | = \ldots 
	= \left( \frac 1 3 \right)^{n-1} \underbrace{|x_1 - x_0 |}_{=4}
\end{align*}
Und für $m \geq n \geq 1$ folgt mit der Dreiecksungleichung:
\begin{align*}
|x_m - x_n| &= |x_m - x_{m-1} + x_{m-1} - x_{m-2} \pm \ldots + x_{n+1} - x_n| \\
	&\leq | x_m - x_{m-1} | + | x_{m-1} - x_{m-2}| + \ldots + |x_{n+1} - x_n| \\
	&= 4 \left( \frac 1 3 \right)^{m-1} + 4 \left( \frac 1 3 \right)^{m-2} + \ldots + 
	4 \left( \frac 1 3 \right)^{n} \\
	&= 4 \left( \frac 1 3 \right)^{n} \left[ 1 + \frac 1 3 + \ldots 
	+ \left( \frac 1 3 \right)^{m-n-1} \right] \\
	&\leq 4 \left( \frac 1 3 \right)^n \sum_{k=0}^\infty \left( \frac 1 3 \right)^k
	= 4 \left( \frac 1 3 \right)^n \cdot \frac{1}{1- \frac 1 3}
	= 4 \cdot \left( \frac 1 3 \right)^n \cdot \frac 3 2 
	= 6 \left( \frac 1 3 \right)^n
\end{align*}
Also bildet $(x_n)$ eine Cauchyfolge und da $\mathbb R$ vollständig ist, konvergiert sie.

Wir bestimmen nun ihren Grenzwert.
Für $n \geq 1$ haben wir
\begin{align*}
x_n - x_{n-1} &= \frac 2 3 x_{n-1} + \frac 1 3 x_{n-2} - x_{n-1}
	= - \frac 1 3 \left( x_{n-1} - x_{n-2}\right)
	= \left( - \frac 1 3 \right)^2 \left( x_{n-2} - x_{n-3} \right) \\
	&= \ldots = \left( - \frac 1 3 \right)^{n-1} \left( x_1 - x_0 \right)
	= (-4) \cdot \left( - \frac 1 3 \right)^{n-1} \ .
\end{align*}
Daraus folgt
\begin{align*}
x_n &= x_{n-1} + (-4) \cdot \left( - \frac 1 3 \right)^{n-1} \\
	&= x_{n-2} + (-4) \cdot \left( - \frac 1 3 \right)^{n-2} 
	+ (-4) \cdot \left( - \frac 1 3 \right)^{n-1} \\
	&= \ldots = x_0 + (-4) \cdot \sum_{k=0}^{n-1} \left( - \frac 1 3 \right)^k
	= 5 - 4 \sum_{k=0}^{n-1} \left( - \frac 1 3 \right)^k \ .
\end{align*}
Der Grenzübergang $n \to \infty$ liefert nun
\begin{align*}
\lim_{n \to \infty} x_n
	&= 5 - 4 \sum_{k=0}^\infty \left( - \frac 1 3 \right)^k
	= 5 - 4 \cdot \frac{1}{1 - \left( - \frac 1 3  \right)}
	= 5 - 4 \cdot \frac{1}{\frac 4 3}
	= 5 - 4 \cdot \frac 3 4 = 2
\end{align*}

\end{proof}



\section{Uneigentliche Konvergenz}
\section{Vollständigkeit}
\section{Reihen}
\section{Absolute Konvergenz}
\section{Potenzreihen}

\setcounter{aufgabe}{1}
\begin{aufgabe}
Die Potenzreihe $a = \sum_k (1+k) X^k$ hat Konvergenzradius $1$ und
für die durch $a$ dargestellte Funktion $\underline a$ gilt: 
$\underline{a}(z) = (1-z)^{-2}$ für $|z| < 1$.
\end{aufgabe}
\begin{proof}
Sei $a_k = 1+k$. Dann ist $a = \sum_k a_k X^k$. Es gilt:
\[
\lim_{k \to \infty} \left| \frac{a_k}{a_{k+1}} \right| = \lim_{k \to \infty}
	\left| \frac{1+k}{2+k} \right| = 1
\]
Also existiert dieser Grenzwert und nach Satz 9.4 ist
\[
\rho_a = \lim_{k \to \infty} \left| \frac{a_k}{a_{k+1}} \right| = 1
\]
der Konvergenzradius von $a$.

Seien $b := \sum_k b_k X^k := \sum_k X^k$ und $c := \sum_k c_k X^k := \sum_k k X^k$.
Diese Reihen haben ebenfalls Konvergenzradius $1$. Also gilt für $z \in \mathbb K$, 
$|z| < 1$:
\[
\underline a(z) = \sum_{k=0}^\infty (1+k) z^k = \sum_{k=0}^\infty z^k + \sum_{k=0}^\infty k z^k
	= \underline b(z) + \underline c(z)
\]
Wir wissen bereits, dass $\underline b(z) = \frac{1}{1-z}$. Wir müssen noch $\underline c(z)$
berechnen. Sei $s_n := \sum_{k=0}^n k z^k$.

\begin{align*}
(1-z) s_n &= (1-z) \sum_{k=0}^n k z^k = \sum_{k=0}^n kz^k - k z^{k+1} \\
					&= \sum_{k=0}^n k z^k - \sum_{k=1}^{n+1} (k-1) z^k
					= 0 + \sum_{k=1}^n ( k z^k - (k-1) z^k ) - n z^{n+1} \\
					&= \sum_{k=0}^{n-1} z^{k+1} - n z^{n+1} = z \sum_{k=0}^{n-1} z^k - n z^{n+1} \\
					&= z \left( \frac{1-z^n}{1-z} \right) - \frac{(1-z) n z^{n+1}}{1-z}
					= \frac{z - z^{n+1} - n z^{n+1} + n z^{n+2}}{1-z} \\
					&= \frac{z - (n+1) z^{n+1} + n z^{n+2}}{1-z}
\end{align*}
Also haben wir $s_n \to \frac{z}{(1-z)^2}$ für $n \to \infty$ und es folgt
\[
\underline c(z) = \sum_{k=0}^\infty k z^k = \frac{z}{(1-z)^2} .
\]
Somit haben wir
\[
\underline a(z) = \underline b(z) + \underline c(z) = \frac{1}{1-z} + \frac{z}{(1-z)^2}
	= \frac{1 -z + z}{(1-z)^2} = \frac{1}{(1-z)^2} .
\]
\end{proof}

\setcounter{aufgabe}{9}
\begin{aufgabe}
	Es sei $b = \sum b_k X^k \in \mathbb C [\![ X ]\!]$ mit 
	$(1-X-X^2)b = 1 \in \mathbb C [\![ X ]\!]$
	\begin{enumerate}
		\item[(a)] Man verifiziere, dass die Koeffizienten $b_k$ die Rekursionsvorschrift
			\[
				b_0 = 1, \quad b_1 = 1, \quad b_{k+1} = b_k + b_{k-1}, \quad k \in \mathbb N^\times \ ,
			\]
			erfüllen, d.h. $(b_k)$ ist die Folge der Fibonacci-Zahlen.
		\item[(b)] Wie gross ist der Konvergenzradius von $b$?
	\end{enumerate}
\end{aufgabe}
\begin{proof}
	\begin{enumerate}
		\item[(a)] Sei $a := 1 - X - X^2$ und $c := 1 \in \mathbb C [\![ X ]\!]$. Aus 
			$1 = c_0 = a_0 b_0 = b_0$ folgt $b_0 = 1$ und aus 
			$0 = c_1 = a_0 b_1 + a_1 b_0 = 1 b_1 + (-1) b_0 = b_1 - 1$ folgt $b_1 = 1$.
			Schliesslich zeigt die Rechnung
			\begin{align*}
				0 &= c_{k+1} = a_0 b_{k+1} + a_1 b_k + a_2 b_{k-1} + \underbrace{a_3 b_{k-2}}_{=0} + 
				\ldots + \underbrace{ a_{k+1} b_0 }_{=0} \\
					&= b_{k+1} - b_k - b_{k-1} \ ,
			\end{align*}
			dass $b_{k+1} = b_k + b_{k-1}$ für $n \in \mathbb N^\times$ gelten muss.
		\item[(b)] Die Folge $(b_n)$ ist genau die Folge der Fibonacci-Zahlen. Nach Aufgabe
			4.9 gilt $\lim_{n \to \infty} \frac{b_{n+1}}{b_n} = g$, wobei $g$ der goldene Schnitt
			bezeichnet. Also folgt mit Satz $9.4$:
			\[
				\rho_b = \lim_{k\to \infty} \left| \frac{b_k}{b_{k+1}} \right|
					= \frac{1}{\displaystyle \lim_{k\to \infty} \left| \frac{b_{k+1}}{b_k}\right|} = \frac 1 g
			\]
	\end{enumerate}
\end{proof}
