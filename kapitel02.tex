\chapter{Konvergenz}
\section{Konvergenz von Folgen}
\section{Das Rechnen mit Zahlenfolgen}
\section{Normierte Vektorräume}
\section{Monotone Folgen}

\setcounter{aufgabe}{3}
\begin{aufgabe}
Für $a \in (0, \infty)$ definiere man die reelle Folge $(x_n)$ rekursiv durch $x_0 \geq a$ und
\[
x_{n+1} := \frac 1 2 \left( x_n + \frac{a}{x_n} \right) \ , \; n \in \mathbb N
\]
Man beweise, dass $(x_n)$ monoton fallend gegen $\sqrt{a}$ konvergiert.
\end{aufgabe}
\begin{proof}
Man weist einfach nach, dass $x_n > 0$ für $n \in \mathbb N$.
Für $n \in \mathbb N$ gilt:
\begin{align*}
x_{n+1}^2 &= \left( \frac 1 2 \left( x_n + \frac{a}{x_n} \right) \right)^2
	= \frac 1 4 \left( x_n^2 + 2a + \frac{a^2}{x_n^2} \right) \\
	&= \frac 1 4 \left( x_n^2 - 2a  + \frac{a^2}{x_n^2} \right) + a 
	= \frac 1 4 \left( x_n - \frac{a}{x_n} \right)^2 + a \geq a
\end{align*}

Wir weisen nach, dass $(x_n)$ monoton fallend ist:
\[
x_{n+1} - x_n = \frac 1 2 \left( x_n + \frac{a}{x_n} \right) - x_n
	= \frac{a}{2x_n} - \frac 1 2 x_n = \frac{a - x_n^2}{2 x_n} \leq 0
\]
Also ist die Folge $(x_n)$ nach unten beschränkt, monoton fallend und konvergiert somit.

Es bleibt noch zu zeigen, dass $\lim x_n = \sqrt{a}$.
Aus 
\[
x_{n+1} - \sqrt{a}
	= \frac 1 2 \left( x_n + \frac{a}{x_n} \right) - \sqrt{a}
	= \frac 1 2 \left( x_n - \sqrt{a} + \frac{a}{x_n} - \sqrt{a} \right)
	= \frac 1 2 \left( 1 - \frac{\sqrt{a}}{x_n} \right) \left( x_n - \sqrt{a} \right)
\]
folgt
\begin{align*}
| x_{n+1} - \sqrt{a} |
	&= \frac 1 2 \underbrace{ \left| 1 - \frac{\sqrt{a}}{x_n} \right| }_{\leq 1}
	\left| x_n - \sqrt{a} \right|
	\leq \frac 1 2 | x_n - \sqrt{a} |
	\leq \ldots \leq \left( \frac 1 2 \right)^{n+1} | x_0 - \sqrt{a} | \ ,
\end{align*}
woraus folgt, dass $x_n \to \sqrt{a}$ für $n \to \infty$.

\end{proof}


\setcounter{aufgabe}{6}
\begin{aufgabe}
\begin{enumerate}
\item[(a)] Man beweise folgende Fehlerabschätzung für $n \in \mathbb N^\times$:
\[
0 < e - \sum_{k=0}^n \frac{1}{k!} < \frac{1}{n n!}
\]
\item[(b)] Man beweise, dass $e$ eine irrationale Zahl ist.
\end{enumerate}
\end{aufgabe}
\begin{proof}
\item[(a)] Da $e = \sum_{k=0}^\infty \frac{1}{k!}$ ist die erste Ungleichung klar.

Sei $y_m := \sum_{k = n+1}^{n + m} \frac{1}{k!}$. Es gilt 
$y_m \to e - \sum_{k=0}^n$ für $m \to \infty$.
\begin{align*}
y_m &= \sum_{k=n+1}^{n+m} \frac{1}{k!}
	= \frac{1}{(n+1)!} \left[ 1 + \frac{1}{n+2} + \frac{1}{(n+2)(n+3)} + \ldots
	+ \frac{1}{(n+2) \cdots (n+m)} \right] \\
	&< \frac{1}{(n+1)!} \left[ 1 + \frac{1}{n+1} + \left( \frac{1}{n+1} \right)^2 + \ldots
	+ \left( \frac{1}{n+1} \right)^{m-2} \right] \\
	&< \frac{1}{(n+1)!} \sum_{k=0}^\infty \left( \frac{1}{n+1} \right)^k
	= \frac{1}{(n+1)!} \frac{1}{1 - \frac{1}{n+1}} 
	= \frac{1}{(n+1)!} \frac{n+1}{n} = \frac{1}{n n!}
\end{align*}
Also gilt auch die zweite Ungleichung.

\item[(b)] Angenommen $e$ ist rational, dann gibt es $p, \ n \in \mathbb N^\times$
mit $e = \frac p n$. Nach (a) gilt dann:
\[
0 < \frac p n - \sum_{k=0}^n \frac{1}{k!} < \frac{1}{n n!}
\]
Also ist
\[
0 < \underbrace{ n! p - n \sum_{k=0}^n \frac{n!}{k!} }_{\in \mathbb Z} < 1 \ .
\]
Das ist aber nicht möglich, da es keine ganze Zahl zwischen 0 und 1 gibt.
\end{proof}

\begin{aufgabe}
Es sei $(x_n)$ rekursiv definiert durch
\[
x_0 := 1, \quad x_{n+1} := 1 + \frac{1}{x_n}, \qquad n \in \mathbb N .
\]
Man zeige, dass die Folge $(x_n)$ konvergiert und bestimme ihren Grenzwert.
\end{aufgabe}
\begin{proof}
Wir zeigen zuerst, dass $x_n > 1$ für $n \geq 1$.
Für $n = 1$ haben wir $x_1 = 1 + \frac 1 1 = 2$.
Nehmen wir an, es gilt $x_n > 1$, so folgt $x_{n+1} = 1 + \frac{1}{x_n} > 1$.
Unmittelbar aus der rekursiven Definition $x_{n+1} = 1 + \frac{1}{x_n}$ und
aus $x_n > 1$ für $n \geq 1$ folgt $x_n < 2$ für $n \geq 2$.
Für $n \geq 1$ gilt sogar $x_n \in \left[ 1.5, \  2 \right]$, da
\[
1.5 = 1 + \frac 1 2 \leq x_{n+1} = 1 + \frac{1}{x_n} \leq 1 + \frac 1 1 = 2 \ .
\]
Insbesondere ist die Folge beschränkt.

Als nächstes zeigen wir, dass die Teilfolge $(x_{2n})$ monoton wachsend ist. Da 
$x_2 = 1 + \frac{1}{x_1} = 1 + \frac{1}{2} = \frac{3}{2}$ ist $x_2 > x_0 = 1$.
Nun sei nach Induktionsannahme $x_{2n} \geq x_{2(n-1)}$.
\begin{align*}
x_{2(n+1)} - x_{2n}
	&= 1 + \frac{1}{x_{2n+1}} - \left( 1 + \frac{1}{x_{2n-1}} \right)
	= \frac{1}{1 + \frac{1}{x_{2n}}} - \frac{1}{1 + \frac{1}{x_{2(n-1)}}}
	= \frac{x_{2n}}{x_{2n}+1} - \frac{x_{2(n-1)}}{x_{2(n-1)} + 1} \\
	&= \frac{ x_{2n} (x_{2(n-1)} + 1) - x_{2(n-1)} (x_{2n} + 1)}{(x_{2n}+1)(x_{2(n-1)}+1)}
	= \frac{x_{2n} - x_{2(n-1)}}{(x_{2n}+1)(x_{2(n-1)}+1)} \geq 0
\end{align*}
Also ist $(x_{2n})$ eine konvergente Teilfolge von $(x_n)$.

Wir weisen nun nach, dass $(x_n)$ eine Cauchyfolge ist. Sei dazu $n \geq 1$ beliebig.
\begin{align*}
| x_{n+1} - x_n | 
	&= \left| 1 + \frac{1}{x_n} - \left( 1 + \frac{1}{x_{n-1}} \right) \right|
	= \left| \frac{ x_{n-1} - x_n}{x_n \cdot x_{n-1}} \right|
	\leq \frac 1 2 | x_{n-1} - x_n | \\
	&\leq \ldots \leq \left( \frac 1 2 \right)^{n-1} \cdot |x_2 - x_1 |
	= \left( \frac 1 2 \right)^n
\end{align*}

Für $m \geq n \geq 1$ erhalten wir
\begin{align*}
| x_m - x_n | 
	&= | x_m - x_{m-1} + x_{m-1} - x_{m-2} \pm \ldots - x_n | \\
	&\leq |x_m - x_{m-1} | + | x_{m-1} - x_{m-2} | + \ldots + |x_{n+1} - x_n| \\
	&\leq \left( \frac 1 2 \right)^{m-1} + \left( \frac 1 2 \right)^{m-2} + \ldots
	+ \left( \frac 1 2 \right)^n \\
	&= \left( \frac 1 2 \right)^n \cdot \sum_{k=0}^{m-1} \left( \frac 1 2 \right)^k 
	\leq \left( \frac 1 2 \right)^n \cdot 2
	= \left( \frac 1 2 \right)^{n-1}
\end{align*}

Somit finden wir zu jedem $\epsilon > 0$ ein $N \in \mathbb N$ so, dass
\[
|x_m - x_n| < \epsilon, \quad \text{für } m \geq n \geq N \ ,
\]
und $(x_n)$ ist eine Cauchyfolge, die eine konvergente Teilfolge besitzt, also selbst
konvergent.

Sei $g \in \left[ 1.5, \ 2 \right]$ der Grenzwert von $(x_n)$.
Mit den Grenzwertsätzen folgt nun
\[
g = \lim_{n \to \infty} x_n = \lim_{n \to \infty} x_{n+1}
	= \lim_{n \to \infty} 1 + \frac{1}{x_n} = 1 + \frac 1 g
\]
Diese Gleichung hat die positive Lösung $g = \frac{{1} + \sqrt{5}}{2}$.
\end{proof}

\begin{proof}[Alternativer Beweis]
Wir zeigen per Induktion $|x_n - g| \leq \frac{1}{g^{n+1}}$, wobei $g$ die positive Lösung der Gleichung
$g = 1 + \frac 1 g $ bezeichnet.

Für $n = 0$ haben wir $|x_0 - g| = | 1 - g | = \left| - \frac 1 g \right| \leq \frac{1}{g^1}$.

Sei nach Induktionsannahme $|x_{n-1} - g | \leq \frac{1}{g^n}$. Dann folgt wegen
$x_n \geq 1$ für alle $n \in \mathbb N$:
\begin{align*}
| x_n - g | 
	&= \left| 1 + \frac{1}{x_{n-1}} - \left( 1 + \frac 1 g \right) \right|
	= \left| \frac{1}{x_{n-1}} - \frac 1 g \right| \\
	&= \left| \frac{g - x_{n-1}}{x_{n-1} \cdot g} \right|
	\leq \frac 1 g \cdot | x_{n-1} - g|
	\leq \frac 1 g \cdot \frac{1}{g^n} = \frac{1}{g^{n+1}}
\end{align*}
Da $g > 1$, folgt $x_n \to g$.
\end{proof}

\begin{aufgabe}
Die \textbf{Fibonacci-Zahlen} $f_n$ sind rekursiv definiert durch
\[
f_0 := 0 \ , \quad f_1 := 1 \ , \quad f_{n+1} := f_n + f_{n-1} \ , \; n \in \mathbb N^\times
\]
Man beweise, dass $\lim \left( \frac{ f_{n+1}}{f_n} \right) = g$, wobei $g$ der
Grenzwert aus Aufgabe 8 bezeichne.
\end{aufgabe}
\begin{proof}
Die Folge der Fibonacci-Zahlen ist monoton wachsend und für $n \geq 1$ gilt $f_n \geq 1$.
Sei $g$ der Grenzwert aus Aufgabe 8, also die positive Lösung der quadratischen Gleichung
$g = 1 + \frac 1 g$.
Sei $F_n := \frac{f_{n+1}}{f_n}$, $n \in \mathbb N^\times$. Wir wollen beweisen, dass
die Folge $(F_n)_{n \geq 1}$ den Grenzwert $g$ hat:
\begin{align*}
| F_n - g |
	&= \left| \frac{ f_{n+1}}{f_n} - g \right|
	= \left| \frac{f_n + f_{n-1}}{f_n} - g \right|
	= \left| 1 + \frac{1}{F_{n-1}} - \left( 1 + \frac 1 g \right) \right| \\
	&= \left| \frac{1}{F_{n-1}} - \frac 1 g \right|
	= \left| \frac{g - F_{n-1}}{ F_{n-1} \cdot g } \right|
	\leq \frac 1 g | F_{n-1} - g | 
	\leq \ldots \leq \left( \frac 1 g \right)^{n-1} |F_1 - g|
\end{align*}
Da $0 < \frac 1 g = g -1 < 1$ folgt
$\left( \frac 1 g \right)^n \to 0$ für $n \to \infty$ und damit
$\lim F_n = g$.
\end{proof}

\begin{aufgabe}
Es seien
\[
x_0 := 5 \ , \quad x_1 := 1 \ , \quad  x_{n+1} := \frac 3 2 x_{n} + \frac 1 3 x_{n-1} \ ,
 \; n \in \mathbb N^\times \ .
\]
Man verifiziere, dass $(x_n)$ konvergiert und bestimme $\lim x_n$.
\end{aufgabe}
\begin{proof}
Für $n \geq 1$ gilt:
\begin{align*}
| x_n - x_{n-1} | &= \left| \frac 2 3 x_{n-1} + \frac 1 3 x_{n-2} - x_{n-1} \right|
	= \frac 1 3 | x_{n-1} - x_{n-2} | = \ldots 
	= \left( \frac 1 3 \right)^{n-1} \underbrace{|x_1 - x_0 |}_{=4}
\end{align*}
Und für $m \geq n \geq 1$ folgt mit der Dreiecksungleichung:
\begin{align*}
|x_m - x_n| &= |x_m - x_{m-1} + x_{m-1} - x_{m-2} \pm \ldots + x_{n+1} - x_n| \\
	&\leq | x_m - x_{m-1} | + | x_{m-1} - x_{m-2}| + \ldots + |x_{n+1} - x_n| \\
	&= 4 \left( \frac 1 3 \right)^{m-1} + 4 \left( \frac 1 3 \right)^{m-2} + \ldots + 
	4 \left( \frac 1 3 \right)^{n} \\
	&= 4 \left( \frac 1 3 \right)^{n} \left[ 1 + \frac 1 3 + \ldots 
	+ \left( \frac 1 3 \right)^{m-n-1} \right] \\
	&\leq 4 \left( \frac 1 3 \right)^n \sum_{k=0}^\infty \left( \frac 1 3 \right)^k
	= 4 \left( \frac 1 3 \right)^n \cdot \frac{1}{1- \frac 1 3}
	= 4 \cdot \left( \frac 1 3 \right)^n \cdot \frac 3 2 
	= 6 \left( \frac 1 3 \right)^n
\end{align*}
Also bildet $(x_n)$ eine Cauchyfolge und da $\mathbb R$ vollständig ist, konvergiert sie.

Wir bestimmen nun ihren Grenzwert.
Für $n \geq 1$ haben wir
\begin{align*}
x_n - x_{n-1} &= \frac 2 3 x_{n-1} + \frac 1 3 x_{n-2} - x_{n-1}
	= - \frac 1 3 \left( x_{n-1} - x_{n-2}\right)
	= \left( - \frac 1 3 \right)^2 \left( x_{n-2} - x_{n-3} \right) \\
	&= \ldots = \left( - \frac 1 3 \right)^{n-1} \left( x_1 - x_0 \right)
	= (-4) \cdot \left( - \frac 1 3 \right)^{n-1} \ .
\end{align*}
Daraus folgt
\begin{align*}
x_n &= x_{n-1} + (-4) \cdot \left( - \frac 1 3 \right)^{n-1} \\
	&= x_{n-2} + (-4) \cdot \left( - \frac 1 3 \right)^{n-2} 
	+ (-4) \cdot \left( - \frac 1 3 \right)^{n-1} \\
	&= \ldots = x_0 + (-4) \cdot \sum_{k=0}^{n-1} \left( - \frac 1 3 \right)^k
	= 5 - 4 \sum_{k=0}^{n-1} \left( - \frac 1 3 \right)^k \ .
\end{align*}
Der Grenzübergang $n \to \infty$ liefert nun
\begin{align*}
\lim_{n \to \infty} x_n
	&= 5 - 4 \sum_{k=0}^\infty \left( - \frac 1 3 \right)^k
	= 5 - 4 \cdot \frac{1}{1 - \left( - \frac 1 3  \right)}
	= 5 - 4 \cdot \frac{1}{\frac 4 3}
	= 5 - 4 \cdot \frac 3 4 = 2
\end{align*}

\end{proof}



\section{Uneigentliche Konvergenz}
\section{Vollständigkeit}
\section{Reihen}
\section{Absolute Konvergenz}
\section{Potenzreihen}

\setcounter{aufgabe}{1}
\begin{aufgabe}
Die Potenzreihe $a = \sum_k (1+k) X^k$ hat Konvergenzradius $1$ und
für die durch $a$ dargestellte Funktion $\underline a$ gilt: 
$\underline{a}(z) = (1-z)^{-2}$ für $|z| < 1$.
\end{aufgabe}
\begin{proof}
Sei $a_k = 1+k$. Dann ist $a = \sum_k a_k X^k$. Es gilt:
\[
\lim_{k \to \infty} \left| \frac{a_k}{a_{k+1}} \right| = \lim_{k \to \infty}
	\left| \frac{1+k}{2+k} \right| = 1
\]
Also existiert dieser Grenzwert und nach Satz 9.4 ist
\[
\rho_a = \lim_{k \to \infty} \left| \frac{a_k}{a_{k+1}} \right| = 1
\]
der Konvergenzradius von $a$.

Seien $b := \sum_k b_k X^k := \sum_k X^k$ und $c := \sum_k c_k X^k := \sum_k k X^k$.
Diese Reihen haben ebenfalls Konvergenzradius $1$. Also gilt für $z \in \mathbb K$, 
$|z| < 1$:
\[
\underline a(z) = \sum_{k=0}^\infty (1+k) z^k = \sum_{k=0}^\infty z^k + \sum_{k=0}^\infty k z^k
	= \underline b(z) + \underline c(z)
\]
Wir wissen bereits, dass $\underline b(z) = \frac{1}{1-z}$. Wir müssen noch $\underline c(z)$
berechnen. Sei $s_n := \sum_{k=0}^n k z^k$.

\begin{align*}
(1-z) s_n &= (1-z) \sum_{k=0}^n k z^k = \sum_{k=0}^n kz^k - k z^{k+1} \\
					&= \sum_{k=0}^n k z^k - \sum_{k=1}^{n+1} (k-1) z^k
					= 0 + \sum_{k=1}^n ( k z^k - (k-1) z^k ) - n z^{n+1} \\
					&= \sum_{k=0}^{n-1} z^{k+1} - n z^{n+1} = z \sum_{k=0}^{n-1} z^k - n z^{n+1} \\
					&= z \left( \frac{1-z^n}{1-z} \right) - \frac{(1-z) n z^{n+1}}{1-z}
					= \frac{z - z^{n+1} - n z^{n+1} + n z^{n+2}}{1-z} \\
					&= \frac{z - (n+1) z^{n+1} + n z^{n+2}}{1-z}
\end{align*}
Also haben wir $s_n \to \frac{z}{(1-z)^2}$ für $n \to \infty$ und es folgt
\[
\underline c(z) = \sum_{k=0}^\infty k z^k = \frac{z}{(1-z)^2} .
\]
Somit haben wir
\[
\underline a(z) = \underline b(z) + \underline c(z) = \frac{1}{1-z} + \frac{z}{(1-z)^2}
	= \frac{1 -z + z}{(1-z)^2} = \frac{1}{(1-z)^2} .
\]
\end{proof}
