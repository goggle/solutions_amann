\chapter{Konvergenz}
\section{Konvergenz von Folgen}
\section{Das Rechnen mit Zahlenfolgen}
\section{Normierte Vektorräume}
\section{Monotone Folgen}
\setcounter{aufgabe}{7}
\begin{aufgabe}
Es sei $(x_n)$ rekursiv definiert durch
\[
x_0 := 1, \quad x_{n+1} := 1 + \frac{1}{x_n}, \qquad n \in \mathbb N .
\]
Man zeige, dass die Folge $(x_n)$ konvergiert und bestimme ihren Grenzwert.
\begin{proof}
Wir zeigen zuerst, dass $x_n > 1$ für $n \geq 1$.
Für $n = 1$ haben wir $x_1 = 1 + \frac 1 1 = 2$.
Nehmen wir an, es gilt $x_n > 1$, so folgt $x_{n+1} = 1 + \frac{1}{x_n} > 1$.
Unmittelbar aus der rekursiven Definition $x_{n+1} = 1 + \frac{1}{x_n}$ und
aus $x_n > 1$ für $n \geq 1$ folgt $x_n < 2$ für $n \geq 2$.
Für $n \geq 1$ gilt sogar $x_n \in \left[ 1.5, \  2 \right]$, da
\[
1.5 = 1 + \frac 1 2 \leq x_{n+1} = 1 + \frac{1}{x_n} \leq 1 + \frac 1 1 = 2 \ .
\]
Insbesondere ist die Folge beschränkt.

Als nächstes zeigen wir, dass die Teilfolge $(x_{2n})$ monoton wachsend ist. Da 
$x_2 = 1 + \frac{1}{x_1} = 1 + \frac{1}{2} = \frac{3}{2}$ ist $x_2 > x_0 = 1$.
Nun sei nach Induktionsannahme $x_{2n} \geq x_{2(n-1)}$.
\begin{align*}
x_{2(n+1)} - x_{2n}
	&= 1 + \frac{1}{x_{2n+1}} - \left( 1 + \frac{1}{x_{2n-1}} \right)
	= \frac{1}{1 + \frac{1}{x_{2n}}} - \frac{1}{1 + \frac{1}{x_{2(n-1)}}}
	= \frac{x_{2n}}{x_{2n}+1} - \frac{x_{2(n-1)}}{x_{2(n-1)} + 1} \\
	&= \frac{ x_{2n} (x_{2(n-1)} + 1) - x_{2(n-1)} (x_{2n} + 1)}{(x_{2n}+1)(x_{2(n-1)}+1)}
	= \frac{x_{2n} - x_{2(n-1)}}{(x_{2n}+1)(x_{2(n-1)}+1)} \geq 0
\end{align*}
Also ist $(x_{2n})$ eine konvergente Teilfolge von $(x_n)$.

Wir weisen nun nach, dass $(x_n)$ eine Cauchyfolge ist. Sei dazu $n \geq 1$ beliebig.
\begin{align*}
| x_{n+1} - x_n | 
	&= \left| 1 + \frac{1}{x_n} - \left( 1 + \frac{1}{x_{n-1}} \right) \right|
	= \left| \frac{ x_{n-1} - x_n}{x_n \cdot x_{n-1}} \right|
	\leq \frac 1 2 | x_{n-1} - x_n | \\
	&\leq \ldots \leq \left( \frac 1 2 \right)^{n-1} \cdot |x_2 - x_1 |
	= \left( \frac 1 2 \right)^n
\end{align*}

Für $m \geq n \geq 1$ erhalten wir
\begin{align*}
| x_m - x_n | 
	&= | x_m - x_{m-1} + x_{m-1} - x_{m-2} \pm \ldots - x_n | \\
	&\leq |x_m - x_{m-1} | + | x_{m-1} - x_{m-2} | + \ldots + |x_{n+1} - x_n| \\
	&\leq \left( \frac 1 2 \right)^{m-1} + \left( \frac 1 2 \right)^{m-2} + \ldots
	+ \left( \frac 1 2 \right)^n \\
	&= \left( \frac 1 2 \right)^n \cdot \sum_{k=0}^{m-1} \left( \frac 1 2 \right)^k 
	\leq \left( \frac 1 2 \right)^n \cdot 2
	= \left( \frac 1 2 \right)^{n-1}
\end{align*}

Somit finden wir zu jedem $\epsilon > 0$ ein $N \in \mathbb N$ so, dass
\[
|x_m - x_n| < \epsilon, \quad \text{für } m \geq n \geq N \ ,
\]
und $(x_n)$ ist eine Cauchyfolge, die eine konvergente Teilfolge besitzt, also selbst
konvergent.

Sei $g \in \left[ 1.5, \ 2 \right]$ der Grenzwert von $(x_n)$.
Mit den Grenzwertsätzen folgt nun
\[
g = \lim_{n \to \infty} x_n = \lim_{n \to \infty} x_{n+1}
	= \lim_{n \to \infty} 1 + \frac{1}{x_n} = 1 + \frac 1 g
\]
Diese Gleichung hat die positive Lösung $g = \frac{{1} + \sqrt{5}}{2}$.
\end{proof}
\end{aufgabe}

\section{Uneigentliche Konvergenz}
\section{Vollständigkeit}
\section{Reihen}
\section{Absolute Konvergenz}
\section{Potenzreihen}

\setcounter{aufgabe}{1}
\begin{aufgabe}
Die Potenzreihe $a = \sum_k (1+k) X^k$ hat Konvergenzradius $1$ und
für die durch $a$ dargestellte Funktion $\underline a$ gilt: 
$\underline{a}(z) = (1-z)^{-2}$ für $|z| < 1$.
\end{aufgabe}
\begin{proof}
Sei $a_k = 1+k$. Dann ist $a = \sum_k a_k X^k$. Es gilt:
\[
\lim_{k \to \infty} \left| \frac{a_k}{a_{k+1}} \right| = \lim_{k \to \infty}
	\left| \frac{1+k}{2+k} \right| = 1
\]
Also existiert dieser Grenzwert und nach Satz 9.4 ist
\[
\rho_a = \lim_{k \to \infty} \left| \frac{a_k}{a_{k+1}} \right| = 1
\]
der Konvergenzradius von $a$.

Seien $b := \sum_k b_k X^k := \sum_k X^k$ und $c := \sum_k c_k X^k := \sum_k k X^k$.
Diese Reihen haben ebenfalls Konvergenzradius $1$. Also gilt für $z \in \mathbb K$, 
$|z| < 1$:
\[
\underline a(z) = \sum_{k=0}^\infty (1+k) z^k = \sum_{k=0}^\infty z^k + \sum_{k=0}^\infty k z^k
	= \underline b(z) + \underline c(z)
\]
Wir wissen bereits, dass $\underline b(z) = \frac{1}{1-z}$. Wir müssen noch $\underline c(z)$
berechnen. Sei $s_n := \sum_{k=0}^n k z^k$.

\begin{align*}
(1-z) s_n &= (1-z) \sum_{k=0}^n k z^k = \sum_{k=0}^n kz^k - k z^{k+1} \\
					&= \sum_{k=0}^n k z^k - \sum_{k=1}^{n+1} (k-1) z^k
					= 0 + \sum_{k=1}^n ( k z^k - (k-1) z^k ) - n z^{n+1} \\
					&= \sum_{k=0}^{n-1} z^{k+1} - n z^{n+1} = z \sum_{k=0}^{n-1} z^k - n z^{n+1} \\
					&= z \left( \frac{1-z^n}{1-z} \right) - \frac{(1-z) n z^{n+1}}{1-z}
					= \frac{z - z^{n+1} - n z^{n+1} + n z^{n+2}}{1-z} \\
					&= \frac{z - (n+1) z^{n+1} + n z^{n+2}}{1-z}
\end{align*}
Also haben wir $s_n \to \frac{z}{(1-z)^2}$ für $n \to \infty$ und es folgt
\[
\underline c(z) = \sum_{k=0}^\infty k z^k = \frac{z}{(1-z)^2} .
\]
Somit haben wir
\[
\underline a(z) = \underline b(z) + \underline c(z) = \frac{1}{1-z} + \frac{z}{(1-z)^2}
	= \frac{1 -z + z}{(1-z)^2} = \frac{1}{(1-z)^2} .
\]
\end{proof}
