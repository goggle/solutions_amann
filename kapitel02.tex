\chapter{Konvergenz}
\section{Konvergenz von Folgen}
\section{Das Rechnen mit Zahlenfolgen}
\section{Normierte Vektorräume}
\section{Monotone Folgen}
\section{Uneigentliche Konvergenz}
\section{Vollständigkeit}
\section{Reihen}
\section{Absolute Konvergenz}
\section{Potenzreihen}

\setcounter{aufgabe}{1}
\begin{aufgabe}
Die Potenzreihe $a = \sum_k (1+k) X^k$ hat Konvergenzradius $1$ und
für die durch $a$ dargestellte Funktion $\underline a$ gilt: 
$\underline{a}(z) = (1-z)^{-2}$ für $|z| < 1$.
\end{aufgabe}
\begin{proof}
Sei $a_k = 1+k$. Dann ist $a = \sum_k a_k X^k$. Es gilt:
\[
\lim_{k \to \infty} \left| \frac{a_k}{a_{k+1}} \right| = \lim_{k \to \infty}
	\left| \frac{1+k}{2+k} \right| = 1
\]
Also existiert dieser Grenzwert und nach Satz 9.4 ist
\[
\rho_a = \lim_{k \to \infty} \left| \frac{a_k}{a_{k+1}} \right| = 1
\]
der Konvergenzradius von $a$.

Seien $b := \sum_k b_k X^k := \sum_k X^k$ und $c := \sum_k c_k X^k := \sum_k k X^k$.
Diese Reihen haben ebenfalls Konvergenzradius $1$. Also gilt für $z \in \mathbb K$, 
$|z| < 1$:
\[
\underline a(z) = \sum_{k=0}^\infty (1+k) z^k = \sum_{k=0}^\infty z^k + \sum_{k=0}^\infty k z^k
	= \underline b(z) + \underline c(z)
\]
Wir wissen bereits, dass $\underline b(z) = \frac{1}{1-z}$. Wir müssen noch $\underline c(z)$
berechnen. Sei $s_n := \sum_{k=0}^n k z^k$.

\begin{align*}
(1-z) s_n &= (1-z) \sum_{k=0}^n k z^k = \sum_{k=0}^n kz^k - k z^{k+1} \\
					&= \sum_{k=0}^n k z^k - \sum_{k=1}^{n+1} (k-1) z^k
					= 0 + \sum_{k=1}^n ( k z^k - (k-1) z^k ) - n z^{n+1} \\
					&= \sum_{k=0}^{n-1} z^{k+1} - n z^{n+1} = z \sum_{k=0}^{n-1} z^k - n z^{n+1} \\
					&= z \left( \frac{1-z^n}{1-z} \right) - \frac{(1-z) n z^{n+1}}{1-z}
					= \frac{z - z^{n+1} - n z^{n+1} + n z^{n+2}}{1-z} \\
					&= \frac{z - (n+1) z^{n+1} + n z^{n+2}}{1-z}
\end{align*}
Also haben wir $s_n \to \frac{z}{(1-z)^2}$ für $n \to \infty$ und es folgt
\[
\underline c(z) = \sum_{k=0}^\infty k z^k = \frac{z}{(1-z)^2} .
\]
Somit haben wir
\[
\underline a(z) = \underline b(z) + \underline c(z) = \frac{1}{1-z} + \frac{z}{(1-z)^2}
	= \frac{1 -z + z}{(1-z)^2} = \frac{1}{(1-z)^2} .
\]
\end{proof}
