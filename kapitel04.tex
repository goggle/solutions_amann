\chapter{Differentialrechnung in einer Variablen}
\section{Differenzierbarkeit}
\setcounter{aufgabe}{8}
\begin{aufgabe}
Es seien $U$ offen in $\mathbb K$, $a \in \mathbb K$ und $f: U \to E$. Man beweise oder widerlege:
\begin{enumerate}
\item[(a)] Ist $f$ in $a$ differenzierbar, so gilt
    \[
         f'(a) = \lim_{h \to 0} \frac{f(a+h) - f(a-h)}{2h}
    \]
\item[(b)] Existiert $\lim_{h \to 0} \frac{f(a+h) - f(a-h)}{2h}$, so ist $f$ in $a$ differenzierbar und es gilt
    \[
        f'(a) = \lim_{h \to 0} \frac{f(a+h) - f(a-h)}{2h}
    \]
\end{enumerate}
\end{aufgabe}
\begin{proof}
    \begin{enumerate}
    \item[(a)]
    Sei $f$ in $a$ differenzierbar, d.h. $f'(a) = \lim_{x \to a} \frac{f(x) - f(a)}{x-a}$ existiert in $E$. Es gilt
    \begin{align*}
        f'(a) &= \lim_{x \to a} \frac{f(x) - f(a)}{x-a} = \lim_{h \to 0} \frac{ f(a+h) - f(a)}{h} \\
        f'(a) &= \lim_{x \to a} \frac{f(x) - f(a)}{x-a} = \lim_{h \to 0} \frac{ f(a) - f(a-h)}{-h} 
            = \lim_{h \to 0} \frac{f(a) - f(a-h)}{h} \ ,
    \end{align*}
    also ist
    \begin{align*}
        f'(a) &= \frac 1 2 \lim_{h \to 0} \frac{ f(a+h)- f(a)}{h} + \frac 1 2 \lim_{h \to 0} \frac{f(a) - f(a-h)}{h} \\
              &= \lim_{h \to 0} \frac{f(a+h) - f(a) + f(a) - f(a-h)}{2h} \\
              &= \lim_{h \to 0} \frac{f(a+h) - f(a-h)}{2h}
    \end{align*}
    \item[(b)]
        Mit $f(x) := |x|$ haben wir 
        \[
            \lim_{h \to 0} \frac{f(0+h) - f(0-h)}{2h} = \lim_{h \to 0} \frac{ |h| - |h|}{2h} = \lim_{h \to 0} \frac{0}{2h} = 0 \ .
        \]
        Der Grenzwert existiert also, aber $f$ ist in $0$ nicht differenzierbar, also ist die Aussage falsch.
    \end{enumerate}
\end{proof}

\section{Mittelwertsätze und ihre Anwendung}
\section{Taylorsche Formeln}
\section{Iterationsverfahren}
