\chapter{Stetige Funktionen}
\section{Stetigkeit}
\setcounter{aufgabe}{10}
\begin{aufgabe}
	Man betrachte die Abbildung
	\[
		f: \mathbb R^2 \to \mathbb R, \quad
			(x, y) \mapsto 
			\begin{cases}
				\frac{xy}{x^2 + y^2}, & (x,y) \neq (0,0) \\
				0, & (x,y) = (0,0)
			\end{cases}
	\]
	und setze für ein festes $x_0 \in \mathbb R$:
	\[
		f_1: \mathbb R \to \mathbb R, \quad x \mapsto f(x, x_0), \qquad 
		f_2: \mathbb R \to \mathbb R, \quad x \mapsto f(x_0, x)
	\]
	Dann gelten:
	\begin{enumerate}
		\item[(a)] $f_1$ und $f_2$ sind stetig.
		\item[(b)] $f$ ist stetig in $\mathbb R^2 \setminus \{(0,0)\}$ und unstetig in
			$(0,0)$.
	\end{enumerate}
\end{aufgabe}
\begin{proof}
	\begin{enumerate}
		\item[(a)] Aus Symmetriegründen reicht es die Stetigkeit von $f_1$ zu beweisen.
			Nehmen wir zuerst $x_0 \neq 0$ an. Dann ist $f_1$ gegeben durch 
			$f_1(x) = f(x, x_0) = \frac{x x_0}{x^2 + x_0^2}$. Da der Nenner nie $0$ ist, 
			ist diese rationale Funktion auf ganz $\mathbb R$ stetig. Im Fall $x_0 = $ ist
			$f_1(x) = \frac{0}{x^2} = 0$ für $x \neq 0$ und $f_1(0) = f(0,0) = 0$, also ist $f_1$
			die Nullfunktion und damit ebenfalls stetig auf $\mathbb R$. Es folgt, dass 
			$f_1$ für jedes feste $x_0 \in \mathbb R$ auf ganz $\mathbb R$ stetig ist.
		\item[(b)] Sei $(z_n) = ( (x_n, y_n) )$ eine Folge in $\mathbb R^2$ mit 
			$\displaystyle \lim_{n \to \infty} z_n = z = (x,y) \neq (0,0)$. Also gilt
			$\displaystyle \lim_{n \to \infty} x_n = x$ und 
			$\displaystyle \lim_{n \to \infty} y_n = y$, also auch
			$\displaystyle \lim_{n \to \infty} x_n^2 = x^2$, 
			$\displaystyle \lim_{n \to \infty} y_n^2 = y^2$, 
			$\displaystyle \lim_{n \to \infty} x_n y_n = xy$ und
			$\displaystyle \lim_{n \to \infty} x_n^2 + y_n^2 = x^2 + y^2$.
			Damit folgt
			\[
				\lim_{n \to \infty} f(x_n, y_n) = \lim_{n \to \infty} \frac{ x_n y_n }{x_n^2 + y_n^2}
					= \frac{xy}{x^2 + y^2} \ ,
			\]
			also ist $f$ stetig in $\mathbb R^2 \setminus \{ (0,0 \}$

			Es sei nun $(z_n) = ( (x_n, x_n) )$ eine Folge in $\mathbb R^2$ mit
			$\displaystyle \lim_{n \to \infty} z_n = (0,0)$.
			Dann ist
			\[
				f( x_n, x_n ) = \frac{ x_n x_n }{x_n^2 + x_n^2} = \frac{x_n^2}{2 x_n^2} = \frac 1 2
			\]
			für jedes $n \in \mathbb N$.
			Wir haben $(x_n, x_n) \to (0, 0)$ aber $f(x_n, x_n) \to \frac 1 2 \neq 0$ für 
			$n \to \infty$ und somit ist $f$ in $(0,0)$ nicht stetig.
	\end{enumerate}
\end{proof}

\begin{aufgabe}
	Man zeige, dass jede lineare Abbildung von $\mathbb K^n$ nach $\mathbb K^m$
	Lipschitz-stetig ist.
\end{aufgabe}
\begin{proof}
	Sei $f: \mathbb K^n \to \mathbb K^m$ linear. Da $f$ linear ist, gibt es eine
	Matrix $A = [a_{ij}] \in \mathbb K^{m \times n}$ so, dass $f(x) = A x$ für $x \in \mathbb K^n$.
	Seien $x, y \in \mathbb K^n$. Mit Satz II.3.12 und der Dreiecksungleichung folgt:
	\begin{align*}
		\Vert f(x) - f(y) \Vert
			&\leq \sqrt m \Vert f(x) - f(y) \Vert_\infty
			= \sqrt m \Vert f(x-y) \Vert_\infty 
			= \sqrt m \Vert A(x-y) \Vert_\infty \\
			&= \sqrt m \max_{1 \leq i \leq m} \Big \vert \sum_{j=1}^n a_{ij} (x_j - y_j) \Big \vert \\
			&\leq \sqrt m \max_{1 \leq i \leq m} \sum_{j=1}^n |a_{ij}| | x_j - y_j | \\
			&\leq \sqrt m \max_{1 \leq i \leq m \atop 1 \leq j \leq n} | a_{ij}| \sum_{j=1}^n |x_j - y_j| \\
			&\leq \sqrt m \max_{1 \leq i \leq m \atop 1 \leq j \leq n} |a_{ij}|\cdot  n \Vert x - y \Vert_\infty \\
			&\leq \underbrace{n \sqrt m \max_{1 \leq i \leq m \atop 1 \leq j \leq n} |a_{ij}| }_{=: \alpha} \Vert x -y \Vert
	\end{align*}
	Dies beweist die Lipschitz-Stetigkeit von $f$.
\end{proof}

\setcounter{aufgabe}{13}
\begin{aufgabe}
	Es seien $(E, (\cdot | \cdot) )$ ein Innenproduktraum und $x_0 \in E$. Dann sind die Abbildungen
	\[
		f: E \to \mathbb K, \quad x \mapsto (x | x_0), \qquad
		g: E \to \mathbb K, \quad x \mapsto (x_0 | x)
	\]
	stetig.
\end{aufgabe}
\begin{proof}
	Sei $\tilde x \in E$, $\epsilon > 0$ und $\delta := \frac{\epsilon}{\Vert x_0 \Vert}$. Für
	$x \in E$ mit $\Vert x - \tilde x \Vert < \delta$ folgt mit der Cauchy-Schwarz-Ungleichung
	\begin{align*}
		| f(x) - f(\tilde x) | 
			&= | (x | x_0) - (\tilde x | x_0) |
			= |(x-\tilde x | x_0) | \\
			&\leq \Vert x - \tilde x \Vert \Vert x_0 \Vert < \delta \Vert x_0 \Vert = \epsilon \ ,
	\end{align*}
	also ist $f$ stetig.
	Analog beweist man die Stetigkeit von $g$.
\end{proof}

\setcounter{aufgabe}{15}
\begin{aufgabe}
	Es sei $n \in \mathbb N^\times$ und $A = [a_{jk}] \in \mathbb K^{n \times n}$.
	Die \textbf{Determinante} von $A$, $\det A$, ist gegeben durch
	\[
		\det A = \sum_{\sigma \in \mathsf S_n} (\sign \sigma) a_{1 \sigma(1)} \cdots 
			a_{n \sigma(n)} \ .
	\]
	Man zeige, dass die Abbildung
	\[
		\mathbb K^{n \times n}  \to \mathbb K, \quad A \mapsto \det A
	\]
	stetig ist.
\end{aufgabe}
\begin{proof}
	Durch die Bijektion
	\[
		\mathbb K^{n \times n} \to \mathbb K^{n \cdot n}, \quad
		\begin{bmatrix} a_{11} & \cdots & a_{1n} \\
			\vdots & & \vdots \\
			a_{n1} & \cdots & a_{nn}
		\end{bmatrix}
		\mapsto
		(a_{11}, \ldots, a_{1n}, a_{21}, \ldots a_{nn} )
	\]
	lässt sich $\mathbb K^{n \times n}$ mit $\mathbb K^{n \cdot n}$
	identifizieren.

	Sei
	\[
		pr_\sigma: \mathbb K^{n \times n} \to \mathbb K^n, \quad
		\begin{bmatrix} a_{11} & \cdots & a_{1n} \\
			\vdots & & \vdots \\
			a_{n1} & \cdots & a_{nn}
		\end{bmatrix}
		\mapsto
		(a_{1\sigma(1)}, \ldots, a_{n \sigma(n)}) \ .
	\]
	Diese Abbildung ist für jedes $\sigma \in \mathsf S_n$ linear und somit nach Aufgabe
	12 Lipschitz-stetig, also insbesondere stetig.

	Sei weiter 
	\[
		( \cdot | \bbone ): \mathbb K^n \to \mathbb K, \quad
		(x_1, \ldots, x_n) \mapsto x_1 \cdots x_n
	\]
	Diese Abbildung ist nach Aufgabe 14 ebenfalls stetig.

	Für $A = [a_{jk}] \in \mathbb K^{n \times n}$ haben wir nun
	\[
		\det(A) = \sum_{\sigma \in \mathsf S_n} (\sign \sigma) a_{1 \sigma(1)} \cdots a_{n \sigma(n)}
			= \sum_{ \sigma \in \mathsf S_n} (\sign \sigma) \left( pr_\sigma(A)) | \bbone \right) \ ,
	\]
	und da Kompositionen sowie endliche Linearkombinationen von stetigen Abbildungen stetig sind,
	ist die Determinante stetig.
\end{proof}


\section{Topologische Grundbegriffe}
\setcounter{aufgabe}{11}
\begin{aufgabe}
	Es seien $X$ und $Y$ metrische Räume, $f: X \to Y$ eine Abbildung. Man beweise:
	\[
		f \text{ ist stetig} \quad \Leftrightarrow \quad f(\overline A) \subset \overline{f(A)}, \ A \subset X
	\]
\end{aufgabe}
\begin{proof}
	\begin{itemize}
		\item[\glqq$\Rightarrow$\grqq] Sei $f$ stetig, $A \subset X$ und $x \in \overline A$. Wir zeigen, 
		dass $f(x) \in \overline {f(A)}$ gilt. Dazu muss nachgewiesen werden, dass $f(x)$ ein 
		Berührungspunkt von $f(A)$ ist, d.h. dass jede Umgebung $V$ von $f(x)$ in $Y$ einen nicht
		leeren Durchschnitt mit $f(A)$ hat.

		Sei $V$ eine beliebige Umgebung von $f(x)$ in $Y$. Aufgrund der Stetigkeit von $f$ finden
		wir eine Umgebung $U$ von $x$ in $X$ so, dass $f(U) \subset V$ gilt. Da $x \in \overline A$, 
		gibt es einen Punkt $y \in U$ mit $y \in A$, also $y \in A \cap U$. Es folgt
		$f(y) \in f(A \cap U) \subset f(A) \cap f(U) = f(A) \cap V$, also ist $f(A) \cap V$ nicht leer
		und es gilt $f(x) \in \overline{ f(A) }$.
	\item[\glqq$\Leftarrow$\grqq] Es gelte also $f(\overline A) \subset \overline{f(A)}$ für
		jedes $A \subset X$. Sei $B \subset Y$ abgeschlossen in $Y$ und sei $A := f^{-1}(B)$.
		Wir zeigen, dass $A$ in $X$ abgeschlossen ist, d.h. dass $A = \overline A$ gilt.
		Nach Definition von $A$ ist $f(A) = f(f^{-1}(B)) \subset B$.
		Sei $x \in \overline A$, dann folgt mit der Voraussetzung
		\[
			f(x) \in f(\overline A) \subset \overline{ f(A) } \subset \overline B = B \ ,
		\]
		also ist $x \in f^{-1}(B) = A$.
	\end{itemize}
\end{proof}

\setcounter{aufgabe}{20}
\begin{aufgabe}
	Es seien $X$ ein metrischer Raum und $A \subset X$. Dann gilt
	\begin{enumerate}
		\item[(i)] Ist $A$ vollständig, so ist $A$ abgeschlossen in $X$. Die Umkehrung ist im Allgemeinen
			falsch.
		\item[(ii)] Ist $X$ vollständig, so ist $A$ genau dann vollständig, wenn $A$ in $X$
			abgeschossen ist.
	\end{enumerate}
\end{aufgabe}
\begin{proof}
	\begin{enumerate}
		\item[(i)] Sei $x \in \overline A$. Nach Korollar 2.10 gibt es eine Folge $(x_n)$ in $A$
			mit $x_n \to x$. Insbesondere ist $(x_n)$ eine Cauchyfolge. Da $A$ vollständig ist, 
			liegt der Grenzwert $x$ in $A$. Also ist $A = \overline A$ und $A$ ist deshalb vollständig.
		\item[(ii)] Die Hinrichtung ist klar nach (i). Für die Rückrichtung sei $A$ abgeschlossen
			in $X$ und $(x_n)$ eine Cauchyfolge mit $x_n \in A$. Da $X$ vollständig ist, konvergiert
			$(x_n)$ gegen einen Grenzwert $x \in X$. Da aber $A$ abgeschlossen ist, ist nach
			Satz 2.11 $x \in A$. Also ist $A$ vollständig.
	\end{enumerate}
\end{proof}


\section{Kompaktheit}
\begin{aufgabe}
	Es seien $X_j$, $j = 0,1, \ldots, n$ metrische Räume. Dann ist $X_1 \times \dots \times X_n$ genau
	dann kompakt, wenn jedes $X_j$ kompakt ist.
\end{aufgabe}
\begin{proof}
	Sei $X_1 \times \dots \times X_n$ kompakt und $(x_{jm})_m$ eine Folge in $X_j$. Dann ist
	$(0, \ldots, 0, x_{jm}, 0, \ldots 0)_m$ eine Folge in $X_1 \times \dots \times X_n$ und hat
	aufgrund der Kompaktheit von $X_1 \times \dots \times X_n$ eine konvergente Teilfolge
	$(0, \ldots, 0, x_{jm_k}, 0, \ldots, 0)_k$. Also ist $(x_{jm_k})_k$ eine konvergente Teilfolge
	in $X_j$ und $X_j$ ist somit kompakt.

	Für die Rückrichtung reicht es zu zeigen, dass für zwei kompakte metrische Räume $X$ und $Y$
	$X \times Y$ wieder kompakt ist. Sei $(x_n, y_n)_n$ eine Folge in $X \times Y$. Dann ist
	$(x_n)$ eine Folge in $X$ und $(y_n)$ eine Folge in $Y$. Wegen der Kompaktheit von 
	$X$ hat $(x_n)$ eine konvergente Teilfolge $(x_{n_k})_k$, d.h.
	$\displaystyle \lim_{k \to \infty} x_{n_k} = x \in X$. Aufgrund der Kompaktheit von $Y$ gibt
	es eine konvergente Teilfolge $(y_{n_{k_j}})_j$, d.h.
	$\displaystyle \lim_{j \to \infty} y_{n_{k_j}} = y \in Y$.
	Dann ist aber $(x_{n_{k_j}})_j)$ wiederum eine Teilfolge von $(x_{n_k})_k$ und hat
	denselben Grenzwert $x$. Also ist die Folge $(x_{n_{k_j}}, y_{n_{k_j}})$ eine 
	konvergente Teilfolge von $(x_n, y_n)$ und es folgt, dass $X \times Y$ kompakt ist.
\end{proof}

\begin{aufgabe}
	Es seien $X$ ein kompakter metrischer Raum und $Y$ eine Teilmenge von $X$. Man beweise:
	$Y$ ist genau dann kompakt, wenn $Y$ abgeschlossen ist.
\end{aufgabe}
\begin{proof}
	Die Hinrichtung folgt direkt aus Satz 3.2.

	Für die Rückrichtung nehmen wir an, $Y$ sei abgeschlossen und $(y_n)_n$ eine Folge in $Y$.
	Da $X$ kompakt ist, hat diese Folge iene konvergente Teilfolge $(y_{n_k})_k$ mit Grenzwert
	$y \in X$. Da $Y$ abgeschlossen ist, gilt aber $y \in Y$. Also hat jede Folge in $Y$
	eine konvergente Teilfolge mit Grenzwert in $Y$ und $Y$ ist somit kompakt.
\end{proof}


\section{Zusammenhang}
Im folgenden bezeichne $X$ stets einen metrischen Raum.

\section{Funktionen in $\mathbb R$}
\section{Die Exponentialfunktion und Verwandte}
